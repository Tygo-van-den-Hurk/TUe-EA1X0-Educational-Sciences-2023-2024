\documentclass{article}

\usepackage{amsmath, amsthm, amssymb, amsfonts}
\usepackage{thmtools}
\usepackage{graphicx}
\usepackage{setspace}
\usepackage{geometry}
\usepackage{float}
\usepackage{hyperref}
\usepackage[utf8]{inputenc}
\usepackage[english]{babel}
\usepackage{framed}
\usepackage[dvipsnames]{xcolor}
\usepackage{tcolorbox}
\usepackage{enumitem}
\usepackage{orcidlink}

%| ----------------------------------------------------------------------------------------------------------------- |%

\newcommand{\HRule}[1]{\rule{\linewidth}{#1}}
\newcommand{\OpenDyslexic}[1]{
    {
        \defaultfontfeatures{Mapping=tex-text,Scale=MatchLowercase}
        \setmainfont{OpenDyslexic}
        \fontsize{24}{26}\selectfont\setstretch{0.5}
        #1
    }
}

%| ----------------------------------------------------------------------------------------------------------------- |%

\setstretch{1.2}
\geometry{ textheight=9in, textwidth=5.5in, top=1in, headheight=12pt, headsep=25pt, footskip=30pt }

%| ----------------------------------------------------------------------------------------------------------------- |%

\title{ 
    \normalsize 
    \textsc{} \\
    [2.0cm]
    \HRule{1.5pt} \\
    \LARGE \textbf{
        \uppercase{
            Educational sciences}
    \HRule{2.0pt} \\ 
    [0.6cm] 
    \LARGE{Peer review} 
    \vspace*{
        10\baselineskip}}
}
\date{\today}
\author{\textbf{Tygo van den Hurk}\,\orcidlink{0009-0003-4182-5076}\textit{(1705709)}}

%| ----------------------------------------------------------------------------------------------------------------- |%

\begin{document}
    
    %| ------------------------------------------------------------------------------------------------------------- |%
    %|                                                  Cover Page                                                   |%
    %| ------------------------------------------------------------------------------------------------------------- |%
    
    \maketitle
    \thispagestyle{empty}
    \newpage
    
    %| ------------------------------------------------------------------------------------------------------------- |%
    %|                                               Table of Contents                                               |%
    %| ------------------------------------------------------------------------------------------------------------- |%
    
    \renewcommand{\contentsname}{Inhoudsopgave}
    \tableofcontents
    \thispagestyle{empty}
    \newpage

    %| ------------------------------------------------------------------------------------------------------------- |%

    \section{Persoonlijke achtergrond}
        
        \subsection{Wat ik gekregen heb:}

            {\itshape      
                Ik ben begonnen op basisschool De Koperwiek in Venlo. Hier heb ik 8 jaar lang een erg prettige leeromgeving gehad. De school lag dicht bij mijn huis en de omgeving voelde altijd vertrouwd. Ik ben een van de gelukkigen die kan zeggen dat hij nooit last heeft gehad van pestgedrag. Ik ben me erg bewust van het geluk wat ik hiermee heb gehad, maar de school heeft hier ook zeker een rol in gespeeld. De Koperwiek was een school die veel lette op elke leerling en ook iedereen de aandacht gaf die ze nodig hadden. 
                
                Mijn beide ouders werken in het onderwijs. Mijn moeder werkt in het basisonderwijs en mijn vader in het middelbaar onderwijs als docent Duits. Docent worden is dus iets wat van jongs af aan altijd in mijn hoofd heeft gezeten. Toen ik in groep 7 meneer Sander kreeg als docent, werd het voor mij ook steeds duidelijker wat voor docent ik graag zou willen zijn. Duidelijk over de regels, maar vooral leuk en gezellig met de hele klas. Altijd proberen om iedere leerling te betrekken bij lessen of activiteiten vond ik dan ook erg belangrijk, zowel als leerlingen, als uit docent perspectief. 
                
                Op de middelbare school heb ik meer van dit soort docenten gehad. Het toeval wilde dat al deze docenten het vak wiskunde doceerden. Vandaar dat mijn keuze voor een minor in wiskunde ook niet uit de lucht komt vallen. De docenten wiskunde op mijn middelbare school leken in bepaalde opzichten allemaal op elkaar. Het waren veelal jonge docenten, die om die reden ook makkelijker een connectie met de leerlingen leken te vinden. Met de theorie die ik nu beheers zou men kunnen zeggen dat deze docenten erg voorstander waren van “samen” gedrag. Deze docenten wisten mij, door de les op een leuke manier te geven, steeds opnieuw te motiveren voor het vak wiskunde. Wat uiteindelijk misschien nog belangrijker is, is dat ze me wisten te motiveren om ook docent te worden. U vraagt zich nu waarschijnlijk af, wat die lessen dan zo leuk en motiverend maakten. Dit vind ik een lastige vraag. Het zat zich vaak in de kleine dingen, zoals op de hoogte zijn van de humor “van dat moment”, de connectie met leerlingen omdat ze dichter bij ons zaten qua leeftijd, maar vooral ook het vermogen om de klas één groep te maken. 
                
                Tegenwoordig studeer ik aan de Technische Universiteit in Eindhoven, maar is de innerlijke docent in me nog altijd niet verdwenen. Ik heb het voorrecht om momenteel actief te zijn als zangdocent en als musicaldocent bij Stichting Musical Stella Duce, een stichting waar ik al 14 jaar bij actief ben en waaraan ik veel te danken heb op het gebied van persoonlijke ontwikkeling. Vroeger was ik een jongen die het soms moeilijk vond om in nieuwe situaties voorop te gaan, of om voor grote groepen te staan. Door het spelen van musicals ben ik hier enorm in gegroeid en heb ik die moeilijkheden eigenlijk volledig achter me kunnen laten. Dit heeft er ook voor gezorgd dat ik mezelf heb kunnen ontwikkelen tot docent, en dus dat ik nu actief ben als zangdocent en musicaldocent. Ik heb hierdoor inmiddels al dik 2 jaar kunnen proeven aan het docentschap, weliswaar in een compleet ander vakgebied dan wiskunde. Dit bevalt me uitstekend! Ik geef les aan twee groepen van basisschoolleeftijd en een groep waarvan de leerlingen allemaal op de middelbare school zitten. Dit zorgt ervoor dat ik heb kunnen ervaren hoe het is om leerlingen die op een middelbare school zitten les te geven. Laten we zeggen dat dit erg mooie, maar ook moeilijke situaties kan verzorgen op het gebied van motivatie en klassenmanagement.
                
                Laat ik daar iets verder op ingaan. Klassenmanagement, in mijn vooral bij leeftijden van middelbare scholieren, is niet altijd even makkelijk. Mijn leeftijd verschilt vaak maar relatief weinig van de leerlingen, wat er soms voor zorgt dat de leerlingen denken dat ze makkelijker een weerwoord kunnen bieden. Het kan er echter ook voor zorgen dat ze het juist erg leuk vinden, waardoor er af en toe onrustige situaties ontstaan met veel lachen. Hoewel ik voorstander ben van leuke lessen, en vind dat lachen moet kunnen, ben ik hier wel in veranderd. Ik heb vooral geleerd hoe belangrijk het is om duidelijk te zijn met leerlingen over lesstructuur, maar vooral ook om ze duidelijk te maken welke momenten geschikt zijn voor grapjes en vooral ook welke momenten dat niet zijn. 
                
                In de toekomst hoop ik deze minor met succes af te ronden. Mijn plan na mijn studie is niet om direct het onderwijs in te gaan. Ik zie de minor meer als persoonlijke ontwikkeling, maar ook vooral als mogelijkheid om in het onderwijs actief te worden als mijn werk in Computer Science me niet zou bevallen. Ik hoop ook nog lange tijd actief te zijn als zangdocent en musicaldocent, daar dit me erg veel plezier bezorgt en me ook zeker helpt in mijn ontwikkeling. Dit zorgt er ook voor dat ik met verschillende doelgroepen kan werken, wat ik erg leuk vind. Ik zou enerzijds kunnen werken als wiskunde docent met iets oudere leerlingen, en anderzijds met veelal basisschoolleerlingen die op de momenten van mijn lessen vooral bezig zijn met hun hobby en er erg veel plezier aan beleven.}

        \bigskip

        \subsection{Mijn feedback voor jou:}

            Je beschrijft je onderwijscarrière en je toekomstperspectief heel gedetailleerd en positief. Het is duidelijk dat je een sterke motivatie hebt om docent te worden en dat je al ervaring hebt opgedaan in het lesgeven, wat geweldig is. Je hebt ook benadrukt hoe belangrijk het is om duidelijk te zijn in de les structuur, wat een belangrijke vaardigheid is. Blijf dit enthousiasme en deze focus op je persoonlijke ontwikkeling behouden terwijl je je onderwijsdoelen nastreeft. Het ziet er goed uit!
            
            Je kan opzich nog wel wat dingen beter doen. Hier zijn enkele suggesties om je verhaal nog beter te maken te maken:
            \begin{itemize}
                \item Concretere voorbeelden: Probeer specifieke voorbeelden te geven van situaties of lessen die je hebt geleerd tijdens je tijd als zangdocent en musicaldocent. Hoe heb je omgegaan met uitdagingen en successen?
                \item Toekomstplannen: Hoewel je hebt vermeld dat je de minor als persoonlijke ontwikkeling ziet en wilt blijven werken als zangdocent, zou het nuttig zijn om te beschrijven hoe je deze ervaringen in de toekomst zou willen combineren met het docentschap in wiskunde. Welke specifieke doelgroepen en leeftijdsgroepen zou je graag willen lesgeven?
                \item Reflectie op klassenmanagement: Je hebt benoemd dat je hebt geleerd over klassenmanagement en duidelijkheid in de lesstructuur, maar het zou helpen als je voorbeelden kunt geven van hoe je deze inzichten in de praktijk hebt toegepast en hoe je van plan bent dit in je toekomstige onderwijsloopbaan te benutten.
            \end{itemize}
    
    \newpage
    
    \section{Taak van de docent}

        \subsection{Wat ik van jou gekregen heb:}

            {\itshape              
                Docenten zijn onder leerlingen, maar ook zeker onder ouders, vaak in opspraak. Vaak gaat het dan om zaken die in mijn ogen relatief onbelangrijk zijn, of zaken die niet begrepen worden. Vaak mist het perspectief van de docent. De vraag is dan: wat zijn dan wel belangrijke zaken?
                
                Allereerst wil ik graag benoemen dat ik veel dingen belangrijk vind aan een docent, maar ook vooral dat er vaak verschillende combinaties mogelijk zijn. Ik zie niet één beeld wat elke docent in mijn ogen moet hebben. Juist de variatie in verschillende types docenten is wat het onderwijs zo divers maakt. Gezien de restricties op de visie die ik aan het uitleggen ben, zal ik op twee zaken ingaan die ik over het algemeen bij alle docenten belangrijk vind.
                
                Ik heb getwijfeld om te starten met de taak “het aanleggen van een basis aan kennis”, echter vind ik dit niet één taak op zich. Het aanleggen van een basis gebeurt alleen als leerlingen gemotiveerd worden om ook daadwerkelijk deze basis te gaan opbouwen. Vandaar dat ik op de volgende twee taken kom.
                
                De eerste taak is communicatie en duidelijkheid. Uit mijn ervaring leer ik dat elke leerling behoefte heeft aan duidelijkheid en duidelijke communicatie vanuit de docent. Deze duidelijkheid hoeft mijn inziens niet alleen gegeven te worden aan leerlingen, maar ook zeker aan ouders. Ik hoor vaak uit verschillende hoeken uit het onderwijs dat steeds meer ouders klagen bij een docent. Met de ervaring die ik heb opgedaan en heb meegekregen vanuit verschillende soorten onderwijs, ervaar ik dat dit klagen vaak kan worden voorkomen door een soepele communicatie. Zowel in het musicalonderwijs als in het middelbare onderwijs, merk ik dat de oorzaak van klagen vaak voortkomt uit het niet communiceren met ouders. De leerling vertelt thuis vaak maar de helft van het verhaal, zodat wanneer de ouders op gesprek komen, de ouders vaak overvallen worden met een andere kant van het verhaal die de situatie verandert. Een voorbeeld hiervan is ouders die klagen over resultaten van leerlingen. Wanneer deze ouders op gesprek komen blijkt de leerling dan bijvoorbeeld afspraken niet te zijn nagekomen, of de leerling heeft huiswerk niet gemaakt, of de leerling blijkt in de les niet actief mee te doen. Deze kant is voor ouders vaak nieuw en kan er dus ook voor zorgen dat een dergelijke klacht snel ingetrokken wordt.
                Leerlingen zelf hebben ook behoefte aan communicatie, bijvoorbeeld over wat er van hen verwacht wordt en wanneer dit verwacht wordt. Dit vloeit vooral voort uit mijn eigen ervaring als zangdocent en mijn ervaring als middelbare scholier. Wanneer leerlingen niet (duidelijk) weten wat er van hen verwacht wordt, ligt de drempel een stuk hoger om aan het betreffende vak te werken. Soms komt dit omdat de leerling echt niet weet waar de leerling moet beginnen, maar soms ziet de leerling ook de mogelijkheid om er onderuit te komen met de smoes “het was niet duidelijk”. 
                Ik vind het zelf fijn om te werk te gaan met een duidelijke planning, die ik vervolgens ook deel met mijn leerlingen. Op deze manier houden mijn leerlingen mij aan de planning, maar weten ze zelf ook precies wat er op welk moment verwacht wordt op het gebied van zelfstudie. 
                Tot slot nog het deel communiceren in de les. De manier waarop er gecommuniceerd wordt vind ik ook erg belangrijk. Ik vind het belangrijk dat de leerlingen zich niet bezwaard voelen om de docent op te zoeken met zowel positieve, als negatieve verhalen. De klas moet het leuk vinden om les te krijgen van een docent door de manier van communicatie, maar moet ook duidelijk merken dat de docent de overhand heeft in de les om onrustige momenten te voorkomen.
                
                De tweede taak is motiveren. Een gemiddelde klas bestaat uit veel verschillende leerlingen met allemaal hun eigen mening over het betreffende vak en hun eigen soort motivatie. Het is de kunst van een docent om voor elke leerling de juiste soort motivatie te bevorderen. Een leerling die bijvoorbeeld wiskunde enorm leuk vind, kan baat hebben bij extra diepgang. Dit stimuleert voor de leerling de passie voor het vak wiskunde nog meer en zorgt ervoor dat deze leerling niet verveeld raakt en daardoor de motivatie verliest. Dit is een voorbeeld van intrinsieke motivatie (lecture slides).
                Weer een andere leerling vindt het wiskunde wellicht helemaal niet leuk, maar weet dat het vak nodig is om een diploma te kunnen halen. Dit is dan weer een voorbeeld van extrinsieke motivatie. 
                Uiteindelijk vind ik het belangrijk dat een docent, op welke manier dan ook, de motivatie van een leerling kan bevorderen. De soort motivatie vind ik dan ook niet altijd belangrijk, zolang de leerling maar gemotiveerd is om te behalen wat de leerling dient te behalen. Wat wél belangrijk is, is dat er gekozen wordt voor een soort motivatie die op de lange termijn aanwezig blijft. Motiveren doormiddel van straffen, kan op de korte termijn helpen. Op de lange termijn, bijvoorbeeld wanneer deze leerling een andere docent krijgt, valt de stimulus voor deze motivatie weg en zal dus ook de motivatie deels wegvallen. Als een leerling gemotiveerd wordt door een docent door middel van enthousiasmeren blijft deze motivatie langer aanwezig. Ook het helpen van een leerling door middel van het samen kijken naar doelen in de aankomende jaren en het inzien dat de leerling het vak nou eenmaal nodig heeft, kan voor sommige leerlingen een lang blijvende motivatie zijn.}

        \bigskip

        \subsection{Mijn feedback voor jou:}

            \subsubsection{Positieve feedback:}

                \begin{itemize}
                    \item Je hebt duidelijk nagedacht over de belangrijke taken van een docent en goed beschreven waarom ze belangrijk zijn.
                    \item Je benadrukt het belang van communicatie en duidelijkheid, wat essentieel is voor zowel leerlingen als ouders.
                    \item Je hebt inzichtelijk gesproken over verschillende vormen van motivatie en benadrukt het belang van langdurige motivatie.
                \end{itemize}
                \bigskip
                \subsubsection{Punten ter verbetering:}
                \begin{itemize}
                    \item De tekst kan worden opgesplitst in kortere alinea's voor een betere leesbaarheid.
                    \item Overweeg om voorbeelden toe te voegen om je standpunten te versterken en praktische toepassingen te illustreren.
                    \item Je zou wat specifieker kunnen zijn over hoe docenten deze taken in de praktijk kunnen uitvoeren, zoals communicatietechnieken en motivatiestrategieën.
                \end{itemize}
    
    \newpage
        
    \section{Mijn Ideale les}
        \subsection{Wat ik van jou gekregen heb:}
            {\itshape             
                Het opstellen van een ideale les vind ik lastig, daar ik vind dat er niet één ideale les bestaat. Ik zal daarom uitleggen wat voor mij een voorbeeld kan zijn van een ideale les. Ik blijf van mening dat verschillende omstandigheden om verschillende lesstructuren vragen.
                Voor deze les ga ik uit van een situatie waarin een klas midden in een hoofdstuk zit, halverwege tussen de start van het hoofdstuk en de toets. Het betreffende vak is wiskunde.
                Allereerst neem ik u mee door hoe mijn les er uit zal zien in grote lijnen, daarna zal ik op delen verder in gaan.
                Om de les te starten, start ik met een interactieve manier om voorkennis op te halen en te activeren. Persoonlijk ben ik voorstander van interactieve lesactiviteiten zoals “Kahoot!”. Dit zou dan ook mijn voorkeur zijn om de les mee te starten. 
                Vervolgens neem ik de klas mee door de structuur van de les. Ik leg uit wat we gaan doen en wat we in deze les willen bereiken.
                Na deze doelen, begin ik met de uitleg. De uitleg zal ik starten met een koppeling tussen de voorkennis en het nieuwe onderwerp. Deze uitleg probeer ik niet langer dan 15 minuten te maken.
                Vervolgens mag de klas zelfstandig aan de slag. Terwijl de klas zelfstandig werkt zal ik extra uitleg geven aan de leerlingen die hier behoefte aan hebben. Na deze extra uitleg mag de klas zelfstandig verder werken of samenwerken. Deze lesactiviteit zal ik doortrekken tot 5 minuten voor het einde van de les.
                Dan zijn er nog 5 minuten over tot de leswisseling. Deze 5 minuten gebruik ik graag om nog eens samen met de klas terug te kijken op de lesdoelen en om te kijken of deze ook bereikt zijn. Uiteraard is er dan ook nog tijd voor een enkele korte vraag. Tot slot zal ik de klas vertellen wat we de volgende les zullen behandelen en nog kort aangeven wat er van de leerlingen verwacht wordt tussen deze en de volgende les.
                
                Lesvoorbereiding
                Om deze les succesvol te kunnen geven, zal er ook voorbereiding nodig zijn. Een duidelijke tijdsplanning, zoals hierboven beschreven, is een onderdeel daarvan. Een ander deel is het voorbereiden van materialen. Ik werk graag met een presentatie bij een les, maar ook de bovengenoemde quiz zal voorbereid moeten worden. Ik ben er voorstander van om presentaties één keer goed te maken en vervolgens in volgende jaren aan te passen naar behoefte. Voor het uitvoeren van een quiz is het belangrijk om na te gaan wat de klas tot hun beschikking heeft. Indien de betreffende school werkt met devices is een online quiz de beste optie in mijn ogen, daar dit het makkelijkste voor te bereiden is. Als dit niet het geval is, zal de docent moeten kijken naar opties als bijvoorbeeld blaadjes met kleuren om antwoorden te geven. Ook tekens afspreken met de klas kan een voorbeeld zijn van het aangeven welk antwoord de leerling geeft. Indien er digitaal gewerkt wordt zal de quiz individueel gemaakt worden. Indien er met blaadjes of tekens gewerkt wordt, is dit ideaal om de klas in groepjes te laten werken en overlegmogelijkheden te creëren. Belangrijk is in dit geval wel om op te letten dat de activiteit niet te lang duurt. 
                
                
                Start van de les
                Zoals gezegd, start ik de les graag op een interactieve manier door middel van bijvoorbeeld een “Kahoot!” quiz. Dit is natuurlijk alleen mogelijk indien alle leerlingen met devices van school werken. Dit is voor leerlingen een meta-cognitieve activiteit, aangezien er georiënteerd wordt, maar ook	 omdat er geëvalueerd wordt of de stof van vorige les goed is blijven hangen (vermunt verloop). Deze activiteit kan echter ook gezien worden als een affectieve activiteit. Doordat leerlingen een quiz vaak leuk vinden, worden leerlingen makkelijker gemotiveerd voor de stof en staan ze welwillender tegenover het leren van nieuwe stof. 
                Vanuit een neuropsychologisch standpunt, is deze werkvorm ook zeer positief. Door het competitieve en “leuke” aspect van een quiz, komt er dopamine vrij, wat ervoor zorgt dat ook de stof in de quiz beter wordt opgenomen en de connectie tussen neuronen wordt versterkt (beeldenbrein). Deze leuke manier van leren zorgt ook voor intrinsieke motivatie (RyanDeci).
                Er wordt in deze werkvorm hard gewerkt aan de voorkennis, waardoor het geheugen een belangrijke rol speelt. Dit zorgt ervoor dat Cognitivisme (ertmer) centraal staat bij deze werkvorm.
                Door de quiz niet te moeilijk te maken wordt er snel gewerkt aan het vervullen van twee van de drie basisbehoeftes. Het gaat hierbij om competentie en relatie (ryandeci). De leerling voelt zich competent doordat het veel vragen goed kan beantwoorden. Doordat de leerling met de hele klas en met de docent bezig is, voelt het zich ook betrokken. 
                Het doel van de docent tijdens dit deel van de les op het gebied van interpersoonlijk gedrag, is vooral om het onderdeel “samen” te versterken (Wubbels et al). 
                
                Structuur en lesdoelen
                Dit onderdeel dient kort te zijn. De leerling krijgt een schema te zien of te horen van de structuur van de les, zodat de leerling weet waar het aan toe is. Samen met de leerlingen wordt er gekeken naar de lesdoelen. De leerlingen weten hierdoor nog beter waar ze aan toe zijn en hebben zo meer duidelijkheid verkregen. Zoals besproken in mijn visie op de belangrijkste taken van een docent, vind ik deze duidelijk erg belangrijk. Dit zowel door persoonlijke ervaringen als kennis.
                
                Uitleg
                Één van de twee grootste onderdelen van de les is de uitleg van nieuwe stof. Leerlingen zijn dan bezig met een cognitieve lesactiviteit (vermunt verloop). 
                Cognitivisme staat centraal in dit lesonderdeel. De leerling koppelt nieuwe stof aan voorkennis en heeft de kans om de nieuwe stof goed te structureren (ertmer). De gemiddelde leerling kent de nieuwe stof nog niet, waardoor de leerling vaak behoefte heeft aan duidelijke regulatie vanuit de docent. Deze situatie zorgt dan voor congruentie in de relatie tussen leerling-regulatie en docent-regulatie (Vermunt verloop).  
                Tijdens de uitleg probeert de docent vooral voorkennis te koppelen aan nieuwe stof, om vervolgens de nieuwe stof uit te leggen. Bij het uitleggen van deze nieuwe stof wordt er vooral een basis gelegd, waardoor de leerlingen vervolgens zelf aan de slag kunnen met deze stof. 
                De docent probeert tijdens dit onderdeel van de les om zich te gedragen volgens de combinatie “boven-samen” (wubbels et al). Door het “boven” deel zorgt de docent ervoor dat de docent duidelijk de bovenhand heeft en de les controleert, waardoor er geen onrustige momenten optreden. Door het “samen” deel motiveert de docent de leerlingen en moedigt de docent de leerlingen aan om te leren.
                
                Zelfstandig werken en extra uitleg
                Bij het volgende deel wordt de les als het ware in 2 groepen gesplitst. Leerlingen die de uitleg goed begrepen hebben kunnen zelfstandig aan de slag met de stof. De autonomie van de leerlingen versterkt dan dopamine, waardoor de stof nog beter beheerst wordt. Ook de intrinsieke motivatie van deze leerlingen wordt door de autonomie bevorderd. Deze zaken zorgen ervoor dat Constructivisme (ertmer) centraal staat.
                De leerlingen die ervoor kiezen om mee te doen met extra uitleg, worden door de docent in de goede richting geholpen. Deze leerlingen hebben behoefte aan meer regulatie van de docent, die ook gegeven wordt. De docent probeert in dit deel samen met de leerlingen te ontdekken. Deze factoren samen zorgen voor constructieve frictie (vermunt verloop). 
                Tijdens deze activiteit vertoont de docent nog steeds “boven-samen” gedrag, maar wordt er iets meer geneigd richting de “samen” kant (wubbels et al). Op deze manier worden de leerlingen die meedoen met de uitleg meer gemotiveerd om actief mee te doen en mee te denken.
                Wat ik persoonlijk mooi vind aan deze vorm van lesgeven, is de mogelijkheid voor elke leerling om te kunnen doen waar de leerling op dat moment behoefte aan heeft. Door verschillende opties aan te bieden werkt iedereen op hun eigen niveau. Het aanbieden van verschillende opties en het aanmoedigen van het bereiken van het niveau wat bij een individuele leerling past, noemt men divergente differentiatie (elsinghorst). Er wordt hiermee gezorgd voor een optimale situatie voor alle leerlingen, wat ervoor zorgt dat de verschillen in prestaties en kennis groter worden. Belangrijk hierin is wel dat er op gelet wordt dat elke leerling op zijn minst de basis van de stof begrijpt aan het einde van de les.
                Een tegenhanger van divergente differentiatie is convergente differentiatie. Hiermee wordt geprobeerd te zorgen dat verschillen tussen leerlingen kleiner worden. In mijn ogen is een nadeel hieraan dat de twee uitersten van de klas benadeeld worden in hun ontwikkelijk.
                Wat ik als leerling ervaren heb, is dat het frustrerend kan zijn als men als leerling gedwongen wordt om te luisteren naar uitleg wanneer dit niet nodig is. Een voorbeeld hiervan zijn mijn lessen Duits in het vierde jaar. Het geval was dat ik met nog twee anderen weinig moeite had met Duits, waardoor we hadden afgesproken met de docent dat wij zelfstandig buiten de les mochten werken. De voorwaarde was dat we aan het einde lieten zien wat we gedaan hadden. Echter, doordat de docent de klas niet onder controle had, werd dit na één les opgeschort door dat de rest van de klas zich benadeeld voelde. Vanaf dit moment werden we weer gedwongen om elke les naar de gehele uitleg te luisteren, wat mijn motivatie voor het vak verminderde. Niet zozeer voor het zelfstandig werken, maar ik zag vooral op tegen de lessen omdat ik deze als saai ervaarde.
                
                Zelfstandig werken
                Wanneer de extra uitleg afgelopen is, is er nog de mogelijkheid voor de leerlingen om zelfstandig te werken of om rustig samen te werken. De leerlingen bouwen zelf aan connecties in hun brein en door het samenwerken wordt de behoefte aan relatie vervuld.
                Dit alles maakt dat deze werk vorm het beste past bij Constructivisme (ertmer). 
                Deze werkvorm zorgt er ook voor dat het laatste onderdeel dat de klas nog miste, namelijk autonomie, vervuld wordt. Hierdoor heeft de klas over de gehele les al hun basisbehoeften vervuld (ryandeci). Door het vervullen van deze basisbehoeftes onthoud de leerling de stof makkelijker en heeft de leerling ook meer aandacht voor de stof.
                De meeste leerlingen zitten aan het begin van deze activiteit op het punt dat ze de lesstof begrepen hebben, en willen graag zelf aan de slag. Doordat de docent minder reguleert, zorgt deze combinatie voor congruentie. Wanneer de docent dan een individuele leerling helpt met vragen zorgt dit ook voor constructieve frictie (vermunt verloop). 
                Tijdens dit lesonderdeel is het voor de docent belangrijk om vooral het initiatief aan de klas te laten, waardoor “boven-samen” verandert naar “onder-samen”. Met het “onder” gedrag laat de docent het initiatief aan de leerlingen, zodat de leerlingen vooral zelf kunnen ontdekken (wubbels et al). Wanneer een leerling om hulp vraagt, is het voor de docent belangrijk om hier op een juiste manier mee om te gaan. Het is belangrijk dat de docent de leerling helpt om zelf naar het goede antwoord te komen, in plaats van het antwoord voor te zeggen. Een manier hiervoor is scaffolding. Met scaffolding wordt er eerst gekeken naar wat het probleem is. Vervolgens wordt door de docent gecontroleerd door middel van vragen of dit ook daadwerkelijk is waar het probleem zit. Als dit het geval is, probeert de docent hulp aan te bieden aan de leerling door vooral strategieën uit te leggen of stukje bij beetje de leerling richting het antwoord te helpen. Hierna checkt de docent of de leerling het ook daadwerkelijk begrepen heeft.
                
                Checken van lesdoelen (5 minuten)
                Aan het einde van de les wordt er nog kort gecontroleerd of de leerlingen de lesdoelen ook daadwerkelijk bereikt hebben. Dit kan de docent doen door bijvoorbeeld met de leerlingen in gesprek te gaan over de stof, dit zou het “samen” gedrag van de docent bevorderen (wubbels et al). De docent zou dit ook kunnen doen door middel van bijvoorbeeld stemmen over in welke maten leerlingen denken dat ze de stof begrepen hebben.
                Door nog een korte vraag over de theorie te stellen kan de docent, maar ook de leerling zelf, controleren of de stof goed verwerkt is.
                Om de les af te sluiten geeft de docent nog kort mee aan de klas wat er de volgende les op de planning staat en wat er tussen de lessen door van de leerlingen verwacht wordt.
                
                Algemeen
                Dit is mijn visie op de ideale les. Belangrijk om te onthouden is dat, zoals ik al eerder vermeld heb, er in mijn ogen niet één ideale les bestaat. Vandaar dat ik gekozen heb voor een algemene situatie en een vrij algemene beschrijving. Ik ben ook van mening dat dit soort lessen goed is voor een algemene structuur, maar dat deze lessen eens in de zoveel tijd moeten worden afgewisseld met een compleet andere les. Een voorbeeld hiervan is bijvoorbeeld een les waarin er gekeken wordt naar een film, of een les waarin er de hele les gewerkt wordt aan een escape room. Ik heb ervaren tijdens Onderwijskunde dat een escape room in het algemeen erg goed werkt om leerlingen te motiveren en activeren.}

        \bigskip
            
        \subsection{Mijn feedback voor jou:}
            \subsubsection{Positieve feedback:}
                \begin{itemize}
                \item Je hebt een gedetailleerde beschrijving gegeven van hoe een ideale les er volgens jou uitziet, inclusief de lesactiviteiten, leeromgeving en docentengedrag.
                \item Je hebt verschillende onderwijskundige theorieën en begrippen geïntegreerd in je uitleg, wat aantoont dat je de theorie begrijpt en kunt toepassen.
                \item Je hebt je persoonlijke ervaringen als leerling in de discussie opgenomen om je standpunt te onderbouwen, wat de tekst verrijkt.
            \end{itemize}
            \subsubsection{Punten ter verbetering:}
            \begin{itemize}
                \item Hoewel je een gedetailleerde beschrijving hebt gegeven, kan de tekst nog compacter en overzichtelijker worden gemaakt door sommige delen samen te voegen of korter te formuleren.
                \item Het zou nuttig zijn om concrete voorbeelden te geven van hoe de theorieën die je noemt daadwerkelijk in de praktijk kunnen worden toegepast in de les. Dit zou je argumenten versterken.
                \item Je kunt overwegen om wat meer variatie toe te voegen aan je ideale lesmodel door in te gaan op verschillende vakgebieden of leeftijdsgroepen, om aan te tonen dat je flexibel bent in je aanpak.
            \end{itemize}

    \newpage

    \section{Referencies}
        \subsection{Wat ik van jou gekregen heb:}
            {\itshape      
                \begin{itemize}
                    \item \url{https://beeldenbrein.nl/dopamine-is-nodig-om-te-leren/#:~:text=Dopamine%20is%20een%20neurotransmitter.,dopamine%20kunnen%20we%20niet%20leren.}
                    \item \url{https://essay.utwente.nl/66630/1/Elsinghorst%20R.%20-%20S0126292%20-%20masterscriptie.pdf }
                \end{itemize}
            }

        \bigskip

        \subsection{Mijn feedback voor jou:}
            \subsubsection{Positieve feedback: (sorry kon maar 1 positiefs ding bedenken)}
                \begin{itemize}
                \item Je hebt er linken in staan.
            \end{itemize}
            \subsubsection{Punten ter verbetering:}
            \begin{itemize}
                \item Dit zijn niet genoeg referencies. Probeer er meer te vinden.
                \item De linken hebben niet de juiste format. Probeer de ze tenminste een format te vinden. Zoals "Auteur, title, link" want nu weet je niks
                \item Geef je links een nummer zodat je weet welke link bij welke referencie hoort.
            \end{itemize}

    \newpage

    \section{Conclusie}
        Rik, je zal echt nog even wat moeten gaan zitten. Ik denk niet dat dit genoeg is om een voldoende te halen. Als je een voldoende haalt, is het een 5.5 en ik weet dat je beter kan dan dit. Het is op het moment dat ik dit document krijg: \textit{6 November 13:49} Je hebt nog 10 uur voor de deadline. Dat is in principe nog te fixen
                
\end{document}