\documentclass{article}

\usepackage{amsmath, amsthm, amssymb, amsfonts}
\usepackage{thmtools}
\usepackage{graphicx}
\usepackage{setspace}
\usepackage{geometry}
\usepackage{float}
\usepackage{hyperref}
\usepackage[utf8]{inputenc}
\usepackage[english]{babel}
\usepackage{framed}
\usepackage[dvipsnames]{xcolor}
\usepackage{tcolorbox}
\usepackage{enumitem}
\usepackage{orcidlink}

%| ----------------------------------------------------------------------------------------------------------------- |%

\newcommand{\HRule}[1]{\rule{\linewidth}{#1}}
\newcommand{\OpenDyslexic}[1]{
    {
        \defaultfontfeatures{Mapping=tex-text,Scale=MatchLowercase}
        \setmainfont{OpenDyslexic}
        \fontsize{24}{26}\selectfont\setstretch{0.5}
        #1
    }
}

%| ----------------------------------------------------------------------------------------------------------------- |%

\setstretch{1.2}
\geometry{ textheight=9in, textwidth=5.5in, top=1in, headheight=12pt, headsep=25pt, footskip=30pt }

%| ----------------------------------------------------------------------------------------------------------------- |%

\title{ 
    \normalsize 
    \textsc{} \\
    [2.0cm]
    \HRule{1.5pt} \\
    \LARGE \textbf{
        \uppercase{
            Educational sciences}
    \HRule{2.0pt} \\ 
    [0.6cm] 
    \LARGE{
        Lesanalyse rondom interpersoonlijk leraarsgedrag} 
    \vspace*{
        10\baselineskip}}
}
\date{\today}
\author{
    \textbf{Tygo van den Hurk}\,\orcidlink{0009-0003-4182-5076}\textit{(1705709)} \and
    \textbf{Rik Janssen} \textit{(1662198)}
}

%| ----------------------------------------------------------------------------------------------------------------- |%

\begin{document}
    
    %| ------------------------------------------------------------------------------------------------------------- |%
    %|                                                  Cover Page                                                   |%
    %| ------------------------------------------------------------------------------------------------------------- |%
    
    \maketitle
    \thispagestyle{empty}
    \newpage
    
    %| ------------------------------------------------------------------------------------------------------------- |%
    %|                                               Table of Contents                                               |%
    %| ------------------------------------------------------------------------------------------------------------- |%
    
    \renewcommand{\contentsname}{Inhoudsopgave}
    \tableofcontents
    \thispagestyle{empty}
    \newpage

    %| ------------------------------------------------------------------------------------------------------------- |%

    \section{Verbaal Gedrag}
        \textbf{Observeer de verbale gedragingen van de docenten in de beide 
        fragmenten, in het bijzonder de beloningen en correcties. Hoe worden 
        beloningen en correcties gegeven? Gebruik als basis het schema verbaal 
        gedrag uit de reader interpersoonlijk leraarsgedrag (Den Brok, 2011, blz. 
        55-56; zie ook het formulier dat we hebben gebruikt in de bijeenkomst over 
        interpersoonlijk gedrag).}
        \begin{enumerate}[label=(\alph*)]
            \item \textbf{Welk verbaal gedrag zie je bij beide docenten als beloning?} \\
                Het verbaal gedrag dat als beloning kan worden gezien, omvat 
                uitspraken zoals "Eindelijk," "ZO! Iedereen is still," gezegd door
                de duits docent in het eerste filmpje.
            \item \textbf{Welk belonend gedrag werkte belemmerend en welk gedrag bevorderend?} \\
                Belemmerend belonend gedrag lijkt te zijn dat er eerst frustratie 
                wordt geuit, zoals "Ik zie de boek nog niet, ik zie de schriften nog
                niet!" Bevorderend belonend gedrag omvat positieve bevestigingen 
                zoals "Eindelijk," maar het kan effectiever zijn als het direct 
                wordt gegeven zonder frustratie vooraf.
            \item \textbf{Welk verbaal gedrag zie je bij beide docenten als correctie?} \\
                Het verbaal gedrag als correctie omvat uitspraken zoals "Ik wil geen
                woord meer horen," "Het bevalt me weer niet," en "Nu geen vraggen 
                meer... SSSTT! De aandacht is nu bij mij.". Dit is omdat, de klas 
                een richting op gaat waar de docent het niet mee eens is, of liever
                niet heeft en met deze uitspraken kon ze de aandacht weer op de les
                richten.
            \item \textbf{Welk corrigerend gedrag werkte belemmerend en welk gedrag bevorderend?} \\
                Belemmerend corrigerend gedrag lijkt te zijn dat de docent 
                herhaaldelijk moet wachten op stilte en herhaaldelijk herinnert aan 
                stilte, zoals "Ik wacht even," en "Het kan heel goed zijn dat jullie
                nog een paar minuten langer in de les blijven zitten." Bevorderend 
                corrigerend gedrag kan zijn om duidelijke verwachtingen te stellen 
                en consequent te zijn in het handhaven van discipline, zoals "OKE! 
                Iedereen even still," en "OKE [Naam], stoelen weg. Die heb je niet 
                nodig."
        \end{enumerate}
    \newpage
    
    %| ------------------------------------------------------------------------------------------------------------- |%

    \section{Non-verbaal gedrag}
        \textbf{Observeer ook de non-verbale gedragingen van beide docenten in de fragmenten. Gebruik hiervoor het schema non-verbaal gedrag uit de reader interpersoonlijk leraarsgedrag (Den Brok, 2011, bijlage 1, blz. 51-53; zie ook formulier uit bijeenkomst). Beantwoord de volgende vragen:}
        \begin{enumerate}[label=(\alph*)]
            \item \textbf{Welk non-verbaal gedrag zie je bij beide docenten?}\\ 
            Bij de eerste docent wordt er veel gebruikgemaakt van een krachtige klemtoon en harde intensiteit. Er is ook een duidelijk gesloten lichaamshouding te zien, dit is te zien aan de armen over elkaar. De docent blijft erg veel voor de klas staan. Later in de video is te zien dat de docent spraakondersteunend rondkijkt om de aandacht van de klas bij de les te houden, bijvoorbeeld van het afvragen van woordjes.
            Bij de tweede docent wordt er gebruik gemaakt van langzame en duidelijke taal. Ook zijn er veel pauzes, zoals wanneer de docent wacht tot de klas klaarzit. Deze docent kijkt spraakvervangend rond, hij wacht namelijk zonder te praten tot de klas stil is. Hoewel de docent wel voor de klas blijft staan, heeft hij wel een open houding.
            \item \textbf{Welk non-verbaal gedrag werkte bevorderend?} \\
            De eerste docent lijkt meer aandacht te krijgen op het moment dat ze iets meer de klas in gaat. Op dit moment kijkt ze meer rond en wijst ze onder andere leerlingen aan, waardoor iedereen meer betrokken wordt. Ook spreekt de docent op dit moment een stuk langzamer, wat meer rust lijkt te brengen.
            Bij de tweede docent is duidelijk te merken dat de klas langzaam maar zeker stil wordt door het gebruiken van lange pauzes en duidelijke taal. Ook de open houding van docent lijkt te helpen bij het winnen van het vertrouwen van de leerlingen, gecombineerd met hoe de docent spreekt tegen de klas. Er is duidelijk te zien hoe de docent lacht bij opmerkingen als "Je haar zit fantastisch". Dit maakt de klas iets "welwillender" om stil te worden.
            \item \textbf{Welk non-verbaal gedrag werkte belemmerend?} \\
            Bij de eerste docent is te merken dat de klas zich tegen de docent keert door het geïrriteerde stemgebruik van de docent en door bijvoorbeeld het klappen om het stil te krijgen. Doordat de docent met de armen over elkaar staat worden de leerlingen niet uitgenodigd om op een leuke manier deel te nemen van de les. De docent spreekt erg snel, wat drukte lijkt op te roepen bij de leerlingen.
            Bij de tweede docent is er weinig wat belemmerend werkt. Aan het einde laat de docent lange pauzes vallen tussen zijn "grapjes" door, wat de leerlingen ruimte geeft om weer drukker te worden en te kletsen. Dit zou belemmerend non-verbaal gedrag kunnen zijn, hoewel het op termijn wellicht een band schept met de klas. 
        \end{enumerate}
    \newpage
    
    %| ------------------------------------------------------------------------------------------------------------- |%

    \section{Interpersoonlijk leraarsgedrag}
        \textbf{Bekijk nu opnieuw het verbale en non-verbale gedrag en probeer het te duiden aan de hand van de interpersoonlijke cirkel.}
        \begin{enumerate}[label=(\alph*)]
            \item \textbf{Beschrijf en benoem het samen en tegen-gedrag van de deelnemende personen (dus zowel docent als leerlingen). Vergelijk beide docenten op hun samen en tegen-gedrag en let hierbij op de verbale en non-verbale gedragingen.}\\ 
                Voorbeeld van samen-gedrag: "Jullie hadden een proefwerk gemaakt. Die krijgen jullie vandaag nog niet terug." - De docent probeert de leerlingen gerust te stellen door aan te kondigen dat ze hun proefwerk morgen zullen ontvangen. Dit is een poging om samen-gedrag te creëren door begrip te tonen voor de verwachtingen van de leerlingen. Voorbeeld van tegen-gedrag: "Klas maakt lawaai en protesteert." - Dit is een duidelijk voorbeeld van tegen-gedrag van de leerlingen omdat ze protesteren en lawaai maken als reactie op de mededeling van de docent.
            \item \textbf{Beschrijf en benoem het boven en onder-gedrag van de deelnemende personen (dus zowel docent als leerlingen). Vergelijk beide docenten op hun boven en onder-gedrag en let hierbij op de verbale en non-verbale gedragingen.} \\
                Voorbeeld van boven-gedrag: "OKE! Iedereen even stil." - De docent vertoont boven-gedrag door met een luide stem instructies te geven en duidelijk te maken dat hij de autoriteit heeft om stilte te eisen. Voorbeeld van onder-gedrag: "De tafels blijven staan." - Onder-gedrag wordt getoond door de leerlingen wanneer ze niet onmiddellijk gehoorzamen aan de instructie van de docent om de tafels weg te halen. In deze voorbeelden wordt het gedrag van zowel de docent als de leerlingen geïllustreerd, en de uitleg laat zien waarom elk gedrag als samen, tegen, boven, of onder-gedrag kan worden beschouwd op basis van de interactie tussen de docent en de leerlingen.
            \item \textbf{Wat is je totaalbeeld van de interpersoonlijke stijl van beide docenten? Hoe komt het op je over? Gebruik de filmpjes om het verschil tussen gedrag en stijl te illustreren.} \\
            Bij filmpje 1 lijkt de interpersoonlijke stijl boven-tegen te zijn. De docent maakt veel opmerkingen die erg negatief zijn tegen de klas en deze opmerkingen lijken van nature te komen. De docent kijkt erg geïrriteerd en geeft bijvoorbeeld aan langer door te gaan met de les of dat ze wéér moet wachten. De nadruk op "weer" toont de irritatie aan. Er is te zien dat de docent dit probeert te corrigeren door bijvoorbeeld een alternatieve datum te geven voor het terugkrijgen van de toetsen.
            Bij filmpje 2 lijkt er meer sprake te zijn van een boven-samen interpersoonlijke stijl. De docent laat aan het begin duidelijk in zijn gedrag zien discipline en duidelijkheid te willen, maar zodra de stilte bereikt is komt de interpersoonlijke stijl naar boven die laat zien naast het boven gedrag, ook vriendelijk en grappig met de klas om te kunnen gaan. Dit maakt het samen gedrag. Een voorbeeld hiervan is zijn grapje over de kapper. 
            \item \textbf{Is in deze filmpjes sprake van complementariteit? Op welke manier(en)?} \\
            In filmpje 1 is de complimentariteit wisselend. In het begin is te zien dat de docent bovengedrag probeert te vertonen, maar de leerlingen lijken hier ook boven-gedrag te vertonen. Een aanleiding hiervoor zou kunnen zijn omdat de docent tegelijkertijd ook tegen-gedrag vertoont. Later, zodra de docent begint te overhoren, zijn de leerlingen een stuk gehoorzamer en rustiger. Op dit moment is er dus wél complimentariteit te zien.
            In filmpje 2 is de complimentariteit duidelijker zichtbaar. De docent toont boven-samen gedrag en de leerlingen antwoorden hierop met onder-samen gedrag. Er wordt snel gehoordzaamd aan de docent zijn opdrachten en opmerkingen. 
        \end{enumerate}
    \newpage
    
    %| ------------------------------------------------------------------------------------------------------------- |%

    \section{Reflectie}
        \textbf{In dit gedeelte beschrijf je je algemene reflectie op de docenten die je gezien hebt. In het algemeen is het gemakkelijker om de zwakkere punten van het gedrag in een video of observatie te benoemen; daarom dagen we jullie in deze opdracht uit te letten op de sterke kanten van beide docenten.}
        \begin{enumerate}[label=(\alph*)]
            \item \textbf{Wat zijn de sterke kanten van beide docenten wanneer het gaat om het klassenmanagement?} \\
            \textbf {Rik}: 
                \begin{itemize}
                    \item Docent 1:
                        \begin{itemize}
                            \item De docent geeft duidelijk haar wensen aan en wacht net zo lang tot dit bereikt is. Ze bezwijkt dus niet onder de druk van de klas.
                            \item De docent weet een rumoerige start van de les om te buigen in een goed lopende les waarbij ze iedereen weet te betrekken.
                        \end{itemize}
                    \item Docent 2:
                        \begin{itemize}
                            \item De docent is duidelijk en er is te zien dat de klas weet wat de docent van hen verwacht. De docent kan ook op een leuke manier de les starten na duidelijke instructies.
                        \end{itemize}
                \end{itemize}
            \textbf{Tygo}: 
                \begin{itemize}
                    \item Docent 1:
                        \begin{itemize}
                            \item Sterk in het handhaven van discipline door duidelijke verbale instructies te geven.
                            \item Toont vastberadenheid om de klas tot stilte te brengen, wat een positief effect kan hebben op de lesvoortgang.
                        \end{itemize}
                    \item Docent 2:
                        \begin{itemize}
                            \item Probeert de aandacht van de klas terug te leiden naar de lesstof door vragen te stellen.
                        \end{itemize}
                \end{itemize}
            \item \textbf{Welke leerpunten haal je hier uit voor jezelf?} \\
            \textbf {Rik}: 
                \begin{itemize}
                    \item Van docent 1 leer ik hoe belangrijk het is om vol te houden aan je standpunt voor een klas om je geloofwaardigheid te houden. Hoe lang het ook duurt tot de klas stil is, uiteindelijk heb je er profijt van.
                    \item Van docent 2 leer ik dat een klas het kan waarderen als er duidelijkheid is, maar als er daarna ook luchtig met de klas omgegaan wordt. Een belangrijk leerpunt is echter dat de luchtigheid niet moet omslaan in ongeconcentreerde leerlingen en rumoer. Dit zou verholpen kunnen worden door wellicht minder pauzes te laten vallen en na een luchtige opmerking ook gelijk door te gaan met de les.
                \end{itemize}
            \textbf{Tygo}: 
                \begin{itemize}
                    \item Van docent 1 kan ik leren hoe ik effectiever discipline kan handhaven door duidelijke en vastberaden verbale instructies te geven.
                    \item Van docent 2 vond ik persoonlijk dat hij een beetje te veel over zich heen liet lopen. Als in hij leek niet echt de klas onder controle te hebben.
                \end{itemize}
            \item \textbf{Welke aanbevelingen (minimaal 1 per docent) heb je zelf voor deze docenten in termen van het geanalyseerde gedrag?} \\
            \textbf {Rik}: 
            \begin{itemize}
                    \item Docent 1:
                        \begin{itemize}
                            \item Ik vond deze docent erg streng en geïrriteerd overkomen op de klas. Ik kan me voorstellen dat je als leerling dan niet gelijk goede zin krijgt om lekker mee te gaan doen. Toon dus duidelijkheid, maar keer jezelf niet tegen de klas.
                        \end{itemize}
                    \item Docent 2:
                        \begin{itemize}
                            \item Zodra de klas eindelijk stil is, is er zeker ruimte voor een leuke grappige opmerking, maar zorg ervoor dat dit niet overgaat in rumoer en gefluister, maar schakel gelijk door naar de lesstof en geef leerlingen niet de kans om te gaan kletsen.
                        \end{itemize}
                \end{itemize}
            \textbf{Tygo}: 
                \begin{itemize}
                    \item Docent 1:
                        \begin{itemize}
                            \item Blijf begrip en empathie tonen, ze kwam namelijk veelste streng over, maar zorg ervoor dat het niet tot een gebrek aan discipline leidt.
                        \end{itemize}
                    \item Docent 2:
                        \begin{itemize}
                            \item Hij mag soms wat strenger zijn om de klas meer onder controle te krijgen. Niet dat de klas aan het exploderen was, maar het voelde ook niet onder controle.
                        \end{itemize}
                \end{itemize}
        \end{enumerate}
    \newpage
    
    %| ------------------------------------------------------------------------------------------------------------- |%

\end{document}
