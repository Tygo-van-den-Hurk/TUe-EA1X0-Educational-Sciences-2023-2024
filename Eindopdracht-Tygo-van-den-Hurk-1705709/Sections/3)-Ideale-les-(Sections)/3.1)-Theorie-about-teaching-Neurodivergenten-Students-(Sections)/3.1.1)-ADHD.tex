            % ------------------------------------------------------------------
            \subsubsection{ADHD}

                \paragraph{ADHD: Struggles}\\
                    Hoewel ADHD veel symptomen en struggels heeft gaan we het hier vandaag alleen hebben over de struggles die van toepassing zijn op de les omgeving:
                    \begin{itemize}
                    
                        \item \textbf{Wanorde en problemen met prioriteren}: 
                            Mensen met ADHD kunnen eigenlijk alleen taken doen die hun stimuleren. Als de taak dat niet is, is het bijna onmogelijk om je te concentrenen op deze taak. Een taak is snel stimulerend als het urgent is. Dit is omdat de taak leid tot het loslaten van adrealine. Doordat niet stimulerenende taken eigenlijk pas mogelijk worden zodra ze urgent worden zijn belangrijken taken vaak niet aan de orde, ze zijn namelijke nog niet urgent. Pas wanneer ze urgent worden, worden ze stimulerened genoeg om te kunnen doen. Hierdoor is er vaak een mismatch met wat belangrijk is en met wat urgent is, vandaar de problemen met prioriteren.\emph{\hyperlink{https://www.youtube.com/watch?v=5xbD9t8cM4M}{Deze video}}\cite{ADHD-video-prioriteiten} legt heel goed uit waarom dit zo moeilijk is.
                        
                        \item \textbf{Slechte tijdmanagementvaardigheden}: 
                            omdat mensen met ADHD vaak zitten te wachten — zo voelen niet-stimulerende-taken vaak namelijk -, of juist zitten te hyperfocusen omdat iets wel stimulerened is en de tijd voorbij vliegt. Verschilt de perceptie in hoe snel tijd gaat veel op een dag. Hierdoor is het lastig voor mensen met ADHD om de tijd bij te houden. Daardoor hebben ze vaak een slechte time management. 
                        
                        \item \textbf{Problemen met het concentreren op een taak}: 
                            als een taak niet stimulerend is is het bijna onmogelijk om je te kunnen concentreren op deze taak. Dit is omdat je brein meer stimulatie en dopamine nodig heeft om te kunnen functioneren. Hierdoor raak je snel afgeleid sinds je automatisch met taken die je wel stimuleren. Als een soort van junkie trekt je brein naar dingen die wel de stimulatie geven die het nodig heeft.
                        
                        \item \textbf{Moeite met multitasking}: 
                            ieder brein heeft een bepaalde hoeveelheid werk geheugen. Dit is de hoeveelheid dingen die je in je hoofd kan houden. Bij mensen met ADHD is deze hoeveelheid lager. Dat betekend dat mensen met ADHD minder dingen te gelijker tijd kunnen herinneren en daarmee word multitasken moeilijk dan wel ommogelijk. \emph{\hyperlink{https://www.youtube.com/watch?v=HszXKZO_H18&ab_channel=HowtoADHD}{Deze video}}\cite{ADHD-video-working-memory} legt het heel goed uit
                        
                        \item \textbf{Overmatige activiteit of rusteloosheid}: 
                            omdat mensen met ADHD graag bezig zijn en gestimuleerd worden is het lastig voor ons om soms stil te zitten en rust te nemen.
                        
                        \item \textbf{Slechte planning}: 
                            dit is lastig voor ieder tiener brein, dat weet iedereen. Het probleem ligt alleen omdat mensen met ADHD niet snel kijken naar de toekomst (dit is namelijk niet urgent/stimulerend genoeg) hebben ze nooit de skill ontwikkeld om in de toekomst te kijken, en daarvoor te plannen. 
                        
                        \item \textbf{Problemen met het afronden van taken}: 
                            als taken niet stimuleren genoeg zijn kan het zijn dat het dopamine potje opraakt en het brein ergens anders heen moet om dopamine te halen. Hierdoor kunnen mensen met ADHD soms moeilijk taken afroden.
                        
                        \item \textbf{Emotionele regulatie}: 
                            de skill om emoties te behandelen en terug te komen naar de default state is lastiger voor mensen met ADHD. Dit uit zich in:
                            \begin{itemize}
                                \item intensere emoties
                                \item blijven hangen in die intense emoties
                                \item moeite met het omgaan met stress
                                \item snel voelen van emoties (die vervolgens heel intense zijn). 
                                \item Rejection Sensitivity, meer hierover kun je \hyperlink{https://www.youtube.com/watch?v=ZQ44ynEjsHQ&ab_channel=KatiMorton}{\emph{hier}}\cite{ADHD-video-rejection-sensitivity} zien, of \hyperlink{https://my.clevelandclinic.org/health/diseases/24099-rejection-sensitive-dysphoria-rsd}{\emph{hier}} lezen\cite{ADHD-rejection-sensitivity}.
                            \end{itemize}
                \end{itemize}
                
                % — — — — — — — — — — — — — — — — — — — — — — — — — — — — — — — 
                \bigskip
                \noindent\paragraph{ADHD: Sterktes}\\
                    Gelukkig heeft ADHD ook een hoop sterkte punten. We gaan wel weer alleen in op de sterkte punten die relevant zouden kunnen zijn voor onze hypotetische les die we dadelijk gaan ontwerpen.
                    \begin{itemize}
                    
                        \item \textbf{Hyperfocus}:
                            als je je leerlingen engaged en gestimuleerd kan krijgen is het mogelijk ze in hyper focus te krijgen. Als een leeraar leert hoe hij dit in diens leerlingen kan oproepen kan dat een goed sterkte punt zijn en leiden tot goede resultaten.

                        \item \textbf{ADHD Hardnekkigheid}:
                            mensen met ADHD zijn vaak veel vol houdender dan hun neurotypische counterparts. Dit komt omdat ze vaak meer moeite moeten doen om de zelfde taak te voltooien. Hierdoor geven mensen met ADHD vaak minder snel op, ze zijn namelijk gewend dat het niet meteen lukt.\cite{ADHD-resilience}

                        \item \textbf{Sociaal}: 
                            mensen met ADHD zijn over het algemeen veel socialer dan Neurotypische mensen. 

                        \item \textbf{Creatief}: 
                            dit is waarschijnlijk het meest bekende voordeel. Mensen met ADHD zijn vaak veel creatiever dan mensen zonder.\cite{ADHD-creativity}

                        \item \textbf{Visuele denkers}: 
                        mensen met ADHD denken veel visueler
                        
                    \end{itemize}
                    
                % — — — — — — — — — — — — — — — — — — — — — — — — — — — — — — — 
                \bigskip
                \noindent\paragraph{ADHD: Leer tips}\\
                enkele tips die ik gevonden heb bij het les geven van mensen met ADHD zijn:
                \begin{itemize}
                
                    \item \textbf{Duidelijke Instructies Geven}: 
                        \textit{Bied duidelijke, stapsgewijze instructies en herhaal of herschrijf ze indien nodig}. Visuele hulpmiddelen en schriftelijke instructies kunnen nuttig zijn\cite{ADHD-visual}. ADHD'ers leren vaak met hun rechter brein. Hierdoor zijn ze veel visueler en leren ze beter als ze het zien of doen over het horen. Een ander voorbeeld is omdat mensen met ADHD een kleinere hoeveelheid werk geheugen hebben kan het zijn dat ze het nodig hebben om dingen opnieuw te kunnen lezen. Ik zal een voorbeeld geven waarin dit terug komt. 

                        \smallskip
                    
                        Laten we zeggen dat een persoon zonder ADHD 4 dingen tegelijkertijd kan onthouden, en iemand met ADHD maar 3. Als je dan een vraag zou stellen aan beide leerlingen neem je een plek in beslag in dit werk geheugen. De niet-ADHD'er heeft er nog 3 over en de ADHD'er nog maar 2. Geef ze dan 3 mogelijke andwoorden op deze vraag en je hebt een probleem. Het hoofd van de normale leerling zit vol (1 vraag plus 3 andwoorden is 4, het maximale wat hij kan onthouden). Maar de ADHD leerling? Die is inmiddels de vraag vergeten. Je gaf 'm namelijk een vierde ding om te onthouden terwijl die er maar physiek 3 kan onthouden. Dit is wanneer de mogelijkheid om terug te kunnen lezen belangrijk kan zijn voor ADHD leerlingen.\cite{ADHD-video-working-memory}

                    \item \textbf{Visuele Ondersteuning Gebruiken}: 
                        \textit{Geef visuele schema's, grafieken en diagrammen}. Dit brengt ons terug op punt 1. Omdat ze visuele leerlingen zijn is het vaak handing om: 
                        \begin{itemize}
                            \item \textbf{Belangrijke dingen te kleuren en te markeren}. 
                                Dit help visueel de belangrijkste dingen uit elkaar te houden.
                            \item \textbf{oudere kinderen te leren hoe ze informatie kunnen visualieseren in diagramen}. 
                                Neem een voorbeeld aan een mind map. Deze heb ik persoonlijk altijd een investeering gevonden. Handig, maar wel tijd consumerend. Jongeren kinderen kun je laten teken wat ze aan het leren zijn. Op deze manier moeten ze de tijd en moeite nemen om het visueel voor te stellen. 
                        \end{itemize}
                        
                        Deze goede voorbeelden kwamen van \emph{\hyperlink{https://www.additudemag.com/visual-learner-homework-help/}{additudemag.com}}. Je kan precieze link vinden in the references onder nummer \cite{ADHD-visual}.
                    
                    \item \textbf{Taken Opsplitsen in Kleinere Stappen}: 
                        \textit{Verdeel complexe taken in kleinere, beheersbare stappen om overweldiging te voorkomen}. Omdat het werkgeheugen minder is, is het belangrijk dat je helpt met het eerst opslitsen van het probleem. Hierdoor raakt het hoofd van mensen met ADHD minder vol, en voorkom je een overload. Dit herkent iedereen wel. Dit is wanneer het je allemaal even te veel word en je brein het niet meer doet. Er zijn namelijk te veel dingen die je moet doen en je weet niet meer waar je mee moet beginnen. Deze drempel ligt bij mensen met ADHD lager. Daarom is het belangrijk dat je ze helpt met hoe ze dit zelf kunnen doen.
    
                        \smallskip
    
                        Een andere goede suggestie die gemaakt werd door \emph{\hyperlink{https://www.additudemag.com/visual-learner-homework-help/}{additudemag.com}} (zie reference \cite{ADHD-visual}) was om een \textit{Plan-of-Attack} te maken. Gezien het soms moeilijk kan zijn voor ons alle dingen op een rijtje te kunnen zetten. \textit{bullet list} zijn ook een super goede om toe te passen. Op deze manier kan je snel terug kijken of je niet iets vergeten bent.

                    \item \textbf{Een Zintuigvriendelijke Omgeving Creëren}: 
                        \textit{Creëer een comfortabele klasomgeving door rekening te houden met zintuiglijke gevoeligheden}. Bied zintuiglijke hulpmiddelen of ruimtes voor zelfregulering. Mensen met ADHD kunnen sneller over, of onder prikkeld worden. Het is belangrijk om de juiste hoeveelheid prikkels te bieden aan het brein en dit regelmatig te houden. Hoe ik dit bijvoorbeeld doe is het zelfde nummer op repeat te houden. Omdat dit je brein stimuleerd maar er toch ritme in zit houd je toch een soort van rust in je hoofd. Omdat het het zelfde nummer is word het heel voorspelbaar, en voor mij heel prettig.

                    % \item \textbf{Zelfvertegenwoordiging Aanmoedigen}: 
                    %     \textit{Leer studenten hun behoeften te identificeren en te pleiten voor passende aanpassingen of strategieën}. Zij zijn althans de studenten met ADHD, laat hun feedback geven of de stimulatie te veel, of te weinig is, de taak te groot of te klein is, of er te veel, of niet genoeg chaos is.

                    \item \textbf{Regelmatig Feedback Geven}: 
                        \textit{Geef specifieke, constructieve feedback en lof om positief gedrag en prestaties te versterken}. Mensen met ADHD hebben behoefte aan snelle feedback. Het prettigste is om meteen feedback te krijgen of iets goed of fout was. Hierdoor krijg je meteen de dopamine die ons brein vereist om verder te gaan. Als we te lang moeten wachten kan dit leiden tot een concentratie, of motivatie problemen. Het probleem is namelijk niet stimulerend genoeg.
                    
                \end{itemize}

            