        \subsection{Conclusion}
            Er waren een paar dingen die we hiervan kunnen leren. Een paar sterktes waar we op in kunnen spelen, en een paar zwaktes die we zouden moeten kunnen trainen en vermijden.

            \subsubsection{Problemen}
                Ten eerste, de meeste neurodivergente mensen zijn visueel ingeteld. Sterker nog, nadat ik mijn onderzoek heb gedaan blijkt dat 65\% visuele leerlingen zijn.\cite{Visual-Learners-are-the-most-common} Daarom moet ik zorgen dat mijn les voornamelijk visueel is ingesteld. De meeste lessen worden nu gegeven worden, worden gedaan door middel van een iemand die voor de klas staat en een verhaal verteld. Niet super ideaal. Tweede probleem is dat afhankelijk van welk type neurodivergente leerling je hebt, dat je verschillende levels van stimulatie moet hebben. Dus daar moeten we ook iets aan doen
            
            \subsubsection{Oplossingen}
                Okay sinds het meeste visueel zal moeten zullen we moeten zorgen dat dat in orde is. Een white bord zal het niet worden. Afhankelijk van wat we er op willen zetten we extreem veel tijd kwijt zijn. Als deze tijd te lang is raakt die ene ADHD'er in de hoek afgeleid. Mij zien tekenen is niet stimulerend genoeg. Dus we zullen in de voor bereiding sowieso tijd moeten stoppen in het maken van een power-point voor alle concepten die we willen behandelen. Samen met tekeningen. Het voordeel is wel. Zodra we een concept getekend hebben kunnen we deze visuele concepten recyclen. We kunnen ze namelijk ieder jaar over en over gebruiken. 
                
                \bigskip
                
                \noindent Hier is een lijst van dingen die we kunnen gebruiken om ons voorbereide powerpoint whiteboard zo appealling als mogelijk te maken voor onze visuele leerlingen.
                
                \begin{itemize}
                    \setlength\itemsep{0em}
                    \item Schrijf nieuwe vocabulary op, en kleur ze zodat ze opvallen;
                    \item gebruik het whiteboard efficient. We zullen ons digitale whiteboard al van te voren gemaakt hebben
                    \item Gebruik grafieken en diagrammen.
                    \item Voeg symbolen en beweging toe aan flashcards.
                    \item Speel flashcardspellen.
                    \item Experimenteer met realia.
                    \item Gebruik diavoorstellingen en video's.
                    \item Moedig hen aan om vooraan te zitten.
                    \item Gebruik OpenDyslextic.
                \end{itemize}
                
                \bigskip
                
                \noindent Ten tweede hoe krijgen we iedereens stimulatie levens optimaal? Ik bijvoorbeeld heb — de hele dag door — het zelfde nummer op repeat. Omdat het zelfde nummer is zit er structuur in. Bij de 5e keer weet je exact waar wat zit. Maar het is toch stimulerend. Dit werkt alleen niet voor iedereen. De ander heeft absolute stilte nodig.
                
                \bigskip
                
                \noindent Hiervoor hebben we kop telefoons. Iedereen heeft tegenwoordig kop telefoons met noice-cannelation wat betekend dat ze in stilte kunnen werken, als ze dit willen. Daarom ga ik tegen stelling tot wat andere docenten doen, het gebruik van koptelefoons stimuleren. Het houd leerlingen gefocust en afgezonderd van elkaar. Wanneer ze zelf mogen werken, werken ze dus ook alleen.
                
                \bigskip
                
                \noindent Dan hebben we nog het laatste ding. We moeten voorspelbaar zijn. Voorspelbaar en gestrucureerd. Dan weten ze precies wat ze te wachten staat. Daarom moeten we de les planning op het bord schrijven en de leerlingen uit nodigen mij als docent te verbeteren en mij aan de tijd te houden. 