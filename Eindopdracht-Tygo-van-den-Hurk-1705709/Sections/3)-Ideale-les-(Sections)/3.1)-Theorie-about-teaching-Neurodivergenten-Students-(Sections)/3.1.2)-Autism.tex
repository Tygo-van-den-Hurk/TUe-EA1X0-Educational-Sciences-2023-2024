    
            % ------------------------------------------------------------------
            \subsubsection{Autisme}
               Autisme en ADHD gaan bijna hand in hand. 50\% tot 70\% van de mensen met autisme heeft ook ADHD\cite{ADHD-en-autisme-overlap}. Dit leid er dan ook toe dat beide zelfde zwaktes en sterke punten hebben.\\
    
                \noindent\paragraph{Autisme: Struggles}\\
                    We gaan wederom de zwakke kanten die niet relevant zijn voor het onderwijs weg laten voor de lengte. Dit mag dan wel een passie essay zijn maar we gaan al richting de 10 pagina's en ik ben nog lang niet klaar.
                    \begin{itemize}
                        \item \textbf{social anxity}: 
                            mensen met autisme hebben vaak moeite, vinden het eng om met mensen contact te maken. Persoonlijk denk ik niet dat dit een ding is dat autisme veroorzaakt maar omdat iemand met autisme vaak niet begrepen word door leeftijds genoten, en misschien zelf leeftijds genoten niet begrijpt word er een soort weerstand opgebouwd met interacties met leeftijds genoten. Want als het eerst niet goed ging, waarom zou het dan nu wel goed gaan?

                            \smallskip

                            Ik bedoel hier niet mee dat het even makkelijk is voor iemand met autisme als iemand zonder om een connectie te leggen. Want gezien je letterlijk op een andere manier je brein hebt opgebouwd is dit zeker niet het geval. Maar ik hoop door het onderwijs beter te maken voor mensen met autisme dat dit in de toekomst te minimaliseren. Het zal namelijk niet 100\% gelijk worden. Maar dat hoeft ook kan ook niet, en dat hoeft ook niet. Zolang als het maar beter, en makkelijker word.
                        
                        \item \textbf{Emotionele regulatie}: 
                            vergelijkbaar met ADHD hebben mensen met autisme last van het reguleren van hun emoties. Dit maakt hun emoties, intenser en langer duurend dan hun neurotypische counterparts. Dit uit zich dan ook in verschillende kenmerken:

                            \begin{itemize}
                                \item druk maken om "niks"
                                \item obsessive compulsive behaviour (OCD)
                                \item meltdowns
                                \item phobias
                            \end{itemize}

                            om dit te voorkomen leerd het brein dingen aan om de emoties te voorkomen in plaats van te leren dealen met deze emoties. Voorbeelden hiervan zijn:

                            \begin{itemize}
                                \item hyper-vigilance, of “shell shocked” lijken
                                \item vermijdend gedrag
                                \item routines en weerstand aan veranderingen
                                \item controlerend gedrag
                                \item of in opstand komen tegen gezag
                            \end{itemize}

                        \item \textbf{shut downs}\cite{autisme-shutdowns}: Dit gebeurd wanneer iemand met autisme overloaded raakt.
                        
                    \end{itemize}
                    
                % — — — — — — — — — — — — — — — — — — — — — — — — — — — — — — — 
                \bigskip
                \noindent\paragraph{Autisme: Sterktes}
                    \begin{itemize}
                        
                        \item \textbf{Focused leren}\cite{autisme-upsides}: 
                            vergelijkbaar met ADHD's hyperfocus kan iemand met autisme de wereld kompleet vergeten en opgaan in waar ze mee bezig zijn.
                            
                        \item \textbf{creativiteit en unique perspecitiven}\cite{autisme-upsides}:
                            Kinderen met autisme hebben vaak een unique blik op iets. Gezien hun brein anders is opgebouwd, en anders informatie verwerkt krijg je natuurlijk ook een andere output!

                        \item \textbf{empathie}\cite{autisme-upsides}: 
                            in tegen stelling tot wat andere denken hebben mensen met autisme meer empathie dan hun neurotypische tegenliggers. Deze mythe komt doordat het moeilijk is om neurotypische mensen om de informatie uit de persoon met autisme te halen.

                        \item \textbf{beter geheugen}\cite{autisme-upsides}: 
                            mensen met autisme hebben vaak een heel goed geheugen. Zeker voor details of patronen. Dit is een van de sterkste kanten van iemand met autisme.

                        \item \textbf{oog voor detail}\cite{autisme-upsides}: 
                            niet alleen kunnen mensen met autisme goed details onthouden, het spotten van details gaat ze heel makkelijk af. Ze zien deze veel beter dan hun neurotypische tegenliggers.
                            
                        \item \textbf{Harnekkigheid}\cite{autisme-upsides}:
                            net zoals ADHD'ers zijn mensen met ADHD hardnekkig en geven ze niet zomaar op. Zekere niet als ze een passie hebben voor iets en ze in hyperfocus gaan.
                            
                    \end{itemize}                    
                    
                % — — — — — — — — — — — — — — — — — — — — — — — — — — — — — — — 
                \bigskip
                \noindent\paragraph{Autisme: Les tips}
                    \begin{itemize}
                        \item \textbf{Structuur en Routine}\cite{autisme-teaching-them}:
                            Het is voor mensen mte autisme heel belangrijk om structuur en routine te hebben. Dit houd ze gefocusd op de taak en zorgt dat ze zich geen zorgen hoeven te maken over kleinen dingen. Een voorbeeld is om na ieder vak kinderen hun spullen op te laten ruimen. Dit geeft dan mentaal het einde van het onderwerp aan. Dit houd het klaslokaal schoon, en structuur in ons hoofd.
                            
                        \item \textbf{Vermijd stimulerende factoren}\cite{autisme-teaching-them}:
                            Om te voorkoment dat mensen met autisme over prikkeld raken is het belangrijk om te zorgen dat de stimulatie tot een minimum getrokken word.
                            
                        \item \textbf{zeg zo min mogelijk en schrijf dingen op}\cite{autisme-teaching-them}:
                            mensen met autisme zijn visuele leerlingen\cite{autisme-visual}. Dus net zoals met ADHD helpt het om:
                                \begin{itemize}
                                    \item duidelijke, stapsgewijze instructies te geven. Herhaal of herschrijf ze indien nodig
                                    \item bied visuele hulpmiddelen zoals highlights/markeren, diagramen 
                                    \item geef schriftelijke instructies in plaats van een heel verhaal te vertellen
                                    \item Laat oudere kinderen te leren hoe ze informatie kunnen visualieseren in diagramen.
                                    \item laat jongeren kinderen kun je laten teken wat ze aan het leren zijn. 
                            \end{itemize}
                            zie de hierboven les tips voor ADHD voor meer informatie.
                        
                        \item \textbf{beprek keuzes}\cite{autisme-teaching-them}:
                            Omdat mensen met autisme focusen op details, en heel analytisch denken kan het zijn dat simple keuzes heel moeilijk kunnen zijn. Als je ze bijvoorbeeld een kleur laat kiezen beperk de opties dan tot 3 verschillende kleuren.
                            
                        \item \textbf{creer een-op-een sociale interacties}\cite{autisme-teaching-them}:
                            Door kinderen met autisme de kans te geven om een op een met 

                    \end{itemize}
