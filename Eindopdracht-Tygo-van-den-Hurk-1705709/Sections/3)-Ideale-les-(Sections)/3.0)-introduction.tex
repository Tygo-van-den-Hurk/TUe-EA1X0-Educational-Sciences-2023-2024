    % --------------------------------------------------------------------------
    \section{Mijn Ideale Les}
        \textbf{De ideale les: Hoe ziet een ideale les er volgens jou uit? Bespreek hier welke leer activiteiten de leerlingen uitvoeren en hoe je die als docent wilt aansturen, inclusief hoe je je daarbij opstelt in termen van interpersoonlijk gedrag. Ga ook in op de tijd, leeromgeving, materialen en groeperingsvormen (zie curriculair spinnenweb). Onderbouw deze ideale les: waarom is dit voor jou een ideale les? Gebruik daarvoor de theorie van Onderwijskunde 1 (wat je hebt geleerd over de neuropsychologie, motivatie en leer theorie) en verwijs ook naar deze theorie. Onderbouw ook met je eigen ervaringen. Wat van wat jij zelf hebt meegemaakt maakt dat je dit als een ideale les ziet en waarom? Je kunt deze les opzetten als een doelsysteem, met de onderbouwingen gekoppeld aan de les onderdelen.}
    
        \bigskip
        
        \noindent Deze sectie ga ik iets anders doen dan dat de opdracht is, maar dit doe ik expres, stay with me. Zover als ik begreep van de opdracht moest je je ideale les beschrijven. Één les. Sinds dat mijn toekomst visie iets breder is dan dat de typische leerling een mooie les mag hebben etcetra. Ga ik deze opdracht ook iets breder maken dat dat nodig is. En kies ik er voor om dieper in te gaan op stof die ik voor deze opdracht behandeld heb. 

        \bigskip
        
        \noindent Maar vrees niet, ik kom uiteindelijk waar ik zijn moet. Dus uiteindelijk krijg je waar je om vraagt. We nemen alleen even een in ADHD-fasion wat men een "flinke detour" zou noemen.