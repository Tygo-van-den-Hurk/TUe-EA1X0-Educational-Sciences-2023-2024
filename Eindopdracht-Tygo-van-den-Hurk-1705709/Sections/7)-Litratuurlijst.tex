    % ------------------------------------------------------------------------------
    % Reference and Cited Works
    % ------------------------------------------------------------------------------
    \bibliographystyle{IEEEtran}
    \pagestyle{empty}
    \renewcommand{\refname}{Literatuurlijst}
    %
    %%%%%%%%% boeken
    % AuteurAchternaam, Initialen. (Jaar). TitelBoek (Editie). Stad, Land: Uitgever.
    %
    %%%%%%%%% journalistisch artiekel
    % AuteurAchternaam, Initialen., & AuteurAchternaam, Initialen. (Jaar). Titel van artikel. Titel van Journal, Volume(Issue), Paginanummer(s).
    %
    %%%%%%%%% Web artiekel
    % AuteurAchternaam, Initialen. (Jaar, Dag maand). TitelArtikel 
    %     [EventueelTypeInternetbron]. Geraadpleegd op dag maand jaar, van http://WebPagina
    %
    \begin{thebibliography}{9}
        % -------------------------------- Les stof --------------------------------
        \item[\bigskip\subsection*{Neurotypisch}]
            \bibitem{enthusiasm-creates-motivation}
                Deci, R. R. E. (2000). Self-determination theory and the facilitation of intrinsic motivation social development, and well-being. Geraadpleegd op 25 Oktober 2023, van \url{https://selfdeterminationtheory.org/SDT/documents/2000 RyanDeci SDT.pdf}.
            \bibitem{intrinsic-motivation-is-more-important}
                Goes, L. (2009). De invloed van intrinsieke- en extrinsieke motivatie op schoolprestaties Geraadpleegd op 25 Oktober 2023, van \url{https://studenttheses.uu.nl/bitstream/handle/20.500.12932/4563/0443603%20LCM.Goes.pdf?sequence=1}.
            \bibitem{extrinsic-motivation-results-in-supperfical-learning}
                Leren, V. (n.d.). Leidt verhoogde motivatie tot betere leerresultaten? Geraadpleegd op 25 Oktober 2023, van \url{https://www.voortgezetleren.nl/leidt-verhoogde-motivatie-tot-betere-leerresultaten/}.
            \bibitem{samen-boven-leads-to-better-results}
                den Brok, P. (2011). Interpersoonlijke ontwikkeling van de docent. Geraadpleegd op 25 Oktober 2023, van \url{https://www.scribd.com/document/341217615/Reader-Interpersoonlijk-Den-Brok}
            \bibitem{repeating-leads-to-long-term-memory}
                Gretchen Schmelzer. (11 Januarie 2015). Understanding Learning and Memory: The Neuroscience of Repetition. Geraadpleegd op 23 Oktober 2023, van \url{https://gretchenschmelzer.com/blog-1/2015/1/11/understanding-learning-and-memory-the-neuroscience-of-repetition}
            \bibitem{NAME-ME}
                Vermunt. J.D. (juni 1999). Congruence and friction between learning and teaching. Geraadpleegd op 25 Oktober 2023, van \url{https://www.sciencedirect.com/science/article/abs/pii/S0959475298000280}
            \bibitem{Kids-that-dont-emtion-perform-worse}
                Bart L.J. Looman, Marlies de Jager, Bram Tuk, Simone A. Onrust, Jeroen Lammers, Marion Spijkerman, Goof Buijs. Sociaal-emotionele ontwikkeling in het primair onderwijs: Een doorlopende integrale aanpak? Geraadpleegd op 23 Oktober 2023, van \url{https://www.kennisrotonde.nl/sites/kennisrotonde/files/migrate/291-antwoord-lestijd-vo.pdf}
            \bibitem{games-help}
                DigitalCitizenship. Teaching and learning with games. Geraadpleegd op 23 Oktober 2023, van \url{https://www.digitalcitizenship.nsw.edu.au/articles/teaching-and-learning-with-games}
            \bibitem{more-homework-equals-bad}
                CLIFTON PARKER. B. (10 Maart 2014). Stanford research shows pitfalls of homework. Geraadpleegd op 24 Oktober 2023, van \url{https://news.stanford.edu/2014/03/10/too-much-homework-031014/}
            \bibitem{compliments-how-to}  
                Tremio, K. (13 januari 2017). Effectieve complimenten geven: Hoe doe je dat? Geraadpleegd op 23 Oktober 2023, van \url{https://www.onderwijsvanmorgen.nl/effectieve-complimenten-geven-hoe/}
        \newpage
       
        % ---------------------------------- ADHD ----------------------------------
        \item[\bigskip\subsection*{ADHD}]
            \bibitem{ADHD-behoeftes}
                NHS. (24 December 2021). Living with Attention deficit hyperactivity disorder (ADHD). Geraadpleegd op 25 Oktober 2023, van \url{https://www.nhs.uk/conditions/attention-deficit-hyperactivity-disorder-adhd/living-with/}.
            \bibitem{ADHD-video-working-memory}
                How to ADHD. (22 Juni 2021). Why I Can't Remember Things -- How ADHD Affects Working Memory. Geraadpleegd op 25 Oktober 2023, van \url{https://www.youtube.com/watch?v=HszXKZO_H18&ab_channel=HowtoADHD}.
            \bibitem{ADHD-video-prioriteiten}
                How to ADHD. (28 September 2023). How ADHD Affects Prioritization (And Why Recognizing IBNUs Can Help). Geraadpleegd op 25 Oktober 2023, van \url{https://www.youtube.com/watch?v=5xbD9t8cM4M}.
            \bibitem{ADHD-video-rejection-sensitivity}
                Kati Morton. (21 Jun 2021). What is Rejection Sensitive Dysphoria? Geraadpleegd op 25 Oktober 2023, van \url{https://www.youtube.com/watch?v=ZQ44ynEjsHQ&ab_channel=KatiMorton}.
            \bibitem{ADHD-rejection-sensitivity}
                Cleveland Clinic. (30 Augustus 2022). Rejection Sensitive Dysphoria (RSD). Geraadpleegd op 25 Oktober 2023, van \url{https://my.clevelandclinic.org/health/diseases/24099-rejection-sensitive-dysphoria-rsd}
            \bibitem{ADHD-resilience}
                Sydni Rubio, Hannah Riley, Sina Eißfeller. (10 mei 2022). The gift of resilience: Why ADHD makes us stronger. Geraadpleegd op 26 Oktober 2023, van \url{https://www.getinflow.io/post/adhd-gift-resilience-makes-us-stronger}
            \bibitem{ADHD-creativity}
                Zahavit Paz. Is the ADHD Brain More Creative? Geraadpleegd op 24 Oktober 2023, van \url{https://www.ldrfa.org/does-adhd-enhance-creative-abilities}
            \bibitem{ADHD-visual}
                Bette Fetter. (8 Maart 2021). See It, Learn It: Make Homework Come Alive for Visual Learners. Geraadpleegd op 24 Oktober 2023, van \url{https://www.additudemag.com/visual-learner-homework-help}
            \bibitem{ADHD-classroom}
                CDC. (27 September 2023). ADHD in the Classroom: Helping Children Succeed in School. Geraadpleegd op 24 Oktober 2023, van \url{https://www.cdc.gov/ncbddd/adhd/school-success.html}
            \bibitem{ADHD-Neurobiologie}
                Wikipedia. (12 Augustus 2023). ADHD. Geraadpleegd op 24 Oktober 2023, van \url{https://nl.wikipedia.org/wiki/ADHD#Neurobiologie}.
            \bibitem{ADHD-Bijkomende-problematiek}
                Wikipedia. (12 Augustus 2023). ADHD. Geraadpleegd op 24 Oktober 2023, van \url{https://nl.wikipedia.org/wiki/ADHD#Bijkomende_problematiek}.
            \bibitem{ADHD-Behandeling}
                Wikipedia. (12 Augustus 2023). ADHD. Geraadpleegd op 24 Oktober 2023, van \url{https://nl.wikipedia.org/wiki/ADHD#Behandeling}.
            \bibitem{ADHD-Niet-medicinale_behandelingen}
                Wikipedia. (12 Augustus 2023). ADHD. Geraadpleegd op 24 Oktober 2023, van \url{https://nl.wikipedia.org/wiki/ADHD#Niet-medicinale_behandelingen}.
            \bibitem{ADHD-en-games}
                Wikipedia. (12 Augustus 2023). ADHD. Geraadpleegd op 24 Oktober 2023, van \url{https://nl.wikipedia.org/wiki/ADHD#Serious_games}.
            \bibitem{ADHD-en-autismespectrumstoornissen}
                Wikipedia. (12 Augustus 2023). ADHD. Geraadpleegd op 24 Oktober 2023, van \url{https://nl.wikipedia.org/wiki/ADHD#ADHD_en_autismespectrumstoornissen}.
        \newpage     
        
        % -------------------------------- Autisme ---------------------------------
        \item[\bigskip\subsection*{Autisme}]
            \bibitem{Autisme}
                Wikipedia. (27 okt 2023). Autisme. Geraadpleegd op 27 Oktober 2023, van \url{https://nl.wikipedia.org/wiki/Autisme}.
            \bibitem{Autisme-lesgeven}
                WaterFord. (10 Maart 2021). 30 Activities, Teaching Strategies, and Resources for Teaching Children with Autism. Geraadpleegd op 27 Oktober 2023, van \url{https://www.waterford.org/education/15-activities-teaching-strategies-and-resources-for-teaching-children-with-autism/}.
             \bibitem{ADHD-en-autisme-overlap}
                Camille Hours, Christophe Recasens, Jean-Marc Baleyte. (28 Februari 2022). ASD and ADHD Comorbidity: What Are We Talking About? Geraadpleegd op 27 Oktober 2023, van \url{https://www.ncbi.nlm.nih.gov/pmc/articles/PMC8918663/}
             \bibitem{autisme-shutdowns}
                Luke Aylward. What are autistic shutdowns and why do they happen? Geraadpleegd op 27 Oktober 2023, van \url{https://www.bristolautismsupport.org/autism-autistic-shutdowns/}
            \bibitem{autisme-upsides} 
                Blossom Childrens Center. (25 April 2023). 7 Incredible Benefits of Autism for Autistic Children. Geraadpleegd op 27 Oktober 2023, van \url{https://blossomchildrenscenter.com/2023/04/25/7-incredible-benefits-of-autism-for-autistic-children/}
            \bibitem{autisme-teaching-them} 
                Positive Action Staf. (25 September 2023). 10 Effective Tips for Teaching Children With Autism in 2023. Geraadpleegd op 27 Oktober 2023, van \url{https://www.positiveaction.net/blog/tips-for-teaching-autistic-children}
            \bibitem{autisme-visual} 
                Lisa Jo Rudy. (25 Oktober 2023). Visual Thinking and Autism: How Visual Tools Can Help Autistic People to Learn and Thrive. Geraadpleegd op 27 Oktober 2023, van \url{https://www.verywellhealth.com/visual-thinking-and-autism-5119992}
        
        % ----------------------------- neurodivergent -----------------------------
        \item[\bigskip\subsection*{Neurodivergent (algemeen)}]
            \bibitem{neurodivergent}
                via \url{https://my.clevelandclinic.org/health/symptoms/23154-neurodivergent} gebruik ik voor de onderbouwing dat er mensen bestaan met een van de grond op anders opgebouwd brein.
            \bibitem{neurodivergent-types}
                \url{https://www.forbes.com/health/mind/what-is-neurodivergent/}
            \bibitem{UDL-guidelines}
                \url{https://udlguidelines.cast.org/}
        \newpage        
                
        % -------------------------------- Dyslextia -------------------------------
        \item[\bigskip\subsection*{Dyslextia}]
            \bibitem{Dyslextie-breinen}
                via \url{https://medium.com/the-ascent/on-the-dyslexic-mind-f6ddb43915d} werd uitgelegd hoe mensen met dyslextie andere breinen hebben die informatie anders verwerken. Citaat: "Because the dyslexic mind is wired in a slightly different way than non-dyslexic minds, we process information differently. This makes us really good at some things but it also means we may struggle with other things, especially if the learning process is not adapted to our way of thinking."    
            \bibitem{dyslextia-struggles-and-superpowers}
                Margo. Things Dyslexia Students Struggle With And Their Hidden Superpowers.  Geraadpleegd op 27 Oktober 2023, van \url{https://oxfordspecialisttutors.com/things-dyslexia-students-struggle-with/}
            \bibitem{dyslextia-different-types-of-memory}
                Root. (14 Mei 2022). Why Some Dyslexics Have Trouble Following Instructions. Geraadpleegd op 27 Oktober 2023, van \url{https://www.learningsuccessblog.com/blog/dyslexia/why-some-dyslexics-have-trouble-following-instructions}
            \bibitem{Creativity-and-Dyslexia}
                Marie Lunney. (16 December 2016). Creativity and Dyslexia. Geraadpleegd op 28 Oktober 2023, van \url{https://www.lexercise.com/blog/creativity-and-dyslexia}
        
        % ------------------------------- Being Queer ------------------------------
        \item[\bigskip\subsection*{Queerness}]
            \bibitem{LGBT-vs-CISHET-breinen}
                Mikhail Votinov, Katharina S. Goerlich, Andrei A. Puiu, Elke Smith, Thomas Nickl-Jockschat, Birgit Derntl, Ute Habel. (3 Maart 2021). Brain structure changes associated with sexual orientation. Geraadpleegd op 27 Oktober 2023, van \url{https://www.nature.com/articles/s41598-021-84496-z}.
                
        % --------------------------------- overig ---------------------------------                
        \item[\bigskip\subsection*{Overig}]
            \bibitem{most-important-brain-dev-age}
                First Things First. Brain Development. \url{https://www.firstthingsfirst.org/early-childhood-matters/brain-development/}
            \bibitem{hyperfocus}
                Smitha Bhandari, Stephanie Langmaid. (25 Augusts 2022). Hyperfocus. Geraadpleegd op 27 Oktober 2023, van \url{https://www.webmd.com/add-adhd/hyperfocus-flow}
            \bibitem{Visual-Learners-are-the-most-common}
                Parker. Understanding Different Types of Learning Styles. Geraadpleegd op 27 Oktober 2023, van \url{https://sites.google.com/a/litchfieldschools.org/parker-lhs-lc-9-12/home/understanding-different-types-of-learning-styles}
            \bibitem{1-in-10-pygy}
                CBS. (25 November 2019). Waardering voor zorgverleners. Geraadpleegd op 20 Oktober 2023, van \url{https://www.cbs.nl/nl-nl/longread/statistische-trends/2019/waardering-voor-zorgverleners/1-inleiding}
            \bibitem{social-isolation-affect-a-childs-mental-health-and-development}
                How does social isolation affect a child’s mental health and development? Geraadpleegd op 20 Oktober 2023, van \url{https://www.noisolation.com/research/how-does-social-isolation-affect-a-childs-mental-health-and-development}
            \bibitem{sugar-equals-drugs}
                Sugar and Dopamine: The Link Between Sweets and Addiction. Geraadpleegd op 22 Oktober 2023, van \url{https://wellnessretreatrecovery.com/sugar-and-dopamine-link-sweets-addiction/}
            \bibitem{succesfull-instructions}
                K. Veroy-Grepl. (2023). Successfull instructions. \url{https://www.tue.nl/en/research/researchers/karen-veroy-grepl}
    \end{thebibliography}
    % ------------------------------------------------------------------------------