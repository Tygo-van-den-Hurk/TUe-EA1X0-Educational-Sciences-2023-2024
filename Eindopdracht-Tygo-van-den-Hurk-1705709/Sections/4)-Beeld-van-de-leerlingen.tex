    % --------------------------------------------------------------------------
    \section{Beeld van de Leelingen}
        \textbf{Beeld van leerlingen. Stel je voor dat je leerlingen hoort praten over jou als docent. Wat zou je willen dat ze dan zeggen? Wat zou je zeker niet willen dat ze zeggen? Waarom?}

        \subsection{Wat ik hoop dat ze zeggen}
            Er is een ding waar ik oprecht heel blij van zou worden. Als mijn leerlingen zouden zeggen dat ze enthousist worden van mij. Dat ze het leuk vinden om naar mijn lessen te komen. Niet alleen zou dit betekenen dat leerlingen open staan voor mij en mijn lessen, maar ook dat het leerprocess met mij oprecht leuk is. 

            \bigskip

            \noindent In een van de lessen was er een hele goede Engels docent. — Ik ben zijn naam vergeten. — Toen zijn filmpje begon had ik de gedachte van "Ha engels litratuur docent, dat is nou iets wat ik echt niet zou willen worden. Ik snap het nut daar dus echt niet van." Maar ik kan met alle zekerheid zeggen dat ik daar fout zat. 
            
            \bigskip

            \noindent In het filmpje was hij aan het les geven op een manier die mij zo aansprak. Die docent was aan het voorlezen. Maar je kon het verhaal echt voor je zien. Maar het was oprecht zo leuk om te zoien. Hij leefde echt in het verhaal, je kon de passie zien zeg maar. Dat sprak mij aan. Ik \hyperlink{https://en.wikipedia.org/wiki/Dungeon_Master}{\underline{DM}} voor de lol, en mijn taak daar is om dan aan een groep spelers de situatie te scheten, en een goed verhaal te vertelen waarin zij de hoofdrol spelen. Dus de hoeveelheid passie waarmee hij zijn verhalen vertelde raakte me.

        \subsection{Wat ik absolute hoop dat ze niet zeggen}
            Ik zou echt het laatste zou zijn dat ik van ze zou willen horen is dat mijn leerlingen denken dat mijn lessen geen waarde toevoegen aan hun leerproces en dat ze er echt niet heel willen. Ik vind het echt super belangrijk dat ze zich betrokken voelen en de lesstof begrijpen. Als leerlingen weg willen van mij en mijn lessen zou ik dat echt heel pijnlijk vinden. Ik denk ook niet dat als dat de sfeer is dat ik mijn leerlingen iets geleerd krijg, omdat zij dan niet open staan om bij mij en mijn lessen te zijn. En omdat ik dan niet de zelfde engerie uit kan stralen als ik weet dat het ongewenst is.

    \newpage
