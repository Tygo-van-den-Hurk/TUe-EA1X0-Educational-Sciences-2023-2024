    % --------------------------------------------------------------------------
    \section{Taak van de docent}
        \textbf{Taak van de docent: Wat zouden voor jou de belangrijkste taken van een docent zijn? (denk bijv. aan: voorbereiden vervolgopleiding, enthousiasmeren, opleiden als goede burger, algemene basiskennis aanleren, sociaal-emotionele ontwikkeling, het 'leren leren', 21e eeuwse vaardigheden aanleren etc). Benoem ze niet allemaal, maar bedenk welke twee jij het belangrijkste vindt en beschrijf waarom die twee voor jou centraal staan.}

        \bigskip
        
        \noindent Naar mijn mening is de rol van een docent veelzijdig, maar ik beschouw twee rollen als bijzonder cruciaal. For starters vind ik Enthousiasmeren het aller belangrijkst en met Promotor van Sociaal-Emotionele Ontwikkeling op de tweede plaats.
        
        \subsection{Enthousiasmeren}
            Het belangrijkste uit dit lijstje vind ik \textit{Enthousiasmeren}, by far. Als leerling werd ik niet vaak enthousist gemaakt over de stof en had mede daarom moeite met het beginnen aan de stof, en blijven werken aan de stof. Als leerlingen enthousiast worden gemaakt over de stof is het niet alleen makkelijker om ze aan de stof te laten beginnen, het is ook makkelijker voor hun om er mee bezig te blijven. Dit is omdat leerlingen die enthousiast zijn over een onderwerp intrinsieke motivatie aanmaken.\cite{enthusiasm-creates-motivation}

            \bigskip
        
            \noindent Hoewel zowel intrinsieke als extrinsieke motivatie resultaten positivief beinvloeden, bleek dat voor de prestaties van mensen intrinsieke motivatie het belangrijkste is.\cite{intrinsic-motivation-is-more-important} Niet alleen is het makkelijker dat leerlingen uit zich zelf aan de stof beginnen, en bezig blijven met de stof. Daar boven op maakt het dat ze actief bezig zijn met de stof. Ze nemen het veel sneller en dieper op. Als leerlingen leren omdat ze extrinsiek gemotiveerd zijn nemen ze de stof veel oppervlakkiger op\cite{extrinsic-motivation-results-in-supperfical-learning}. Een ander aspect van leerlingen die extrinsiek gemotiveerd zijn is dat omdat ze het willen leren ze naar je toe komen voor hulp en begeleiding. Hierdoor krijg je dat de sfeer zich automatisch op steld als samen-boven. Wat door zowel leerlingen als leeraren werd gezien als de ideale combinatie. Niet alleen vinden beide dit het beste, ook bleek dat samen-boven leid tot betere prestaties.\cite{samen-boven-leads-to-better-results} Daarom vind ik intrinsieke motivatie veel belangrijker dan extrinsieke motivatie en wil ik die zoveel mogelijk opwekken. 
            % Als kers op de taart omdat leerlingen zich actief op stellen en open staan voor de stof is het makkelijker om ze feedback te geven en te begleiden. Ze zullen hier zelfs actief om vragen
            
            % Bij leerlingen met ADHD werd dit effect dubble op. Mensen met ADHD hebben een dopamine probleem, als we niet enthousist zijn over iets, en het niet urgent is krijgen we niet de dopamine of adrealine die iemand met ADHD nodig heeft om taken af te maken. Wat dus betekend dat als dit enthoustme ontbreekt ze niet alleen de boost missen, maar ook nog eens 
            
        \subsection{Promotor van Sociaal-Emotionele Ontwikkeling}
            Als tweede vind ik sociaal-emotioneel leren super belangrijk. Ik ben van mening dat voor kinderen — en sommige volwassenen — het net zo belangrijk is dat ze Sociaal en Emotioneel capabel zijn als dat ze weten hoe je een keer sommetje maakt. Het feit dat sommige volwassenen nog steeds niet weten hoe je je in sociale momenten je op moet stellen. Of dat ze nog steeds niet uitgevogeld hebben je met je emoties om moet gaan absurd.

            \bigskip
        
            \noindent 1 op de 10 mensen van de Nederlandse bevolking heeft psycholoog, psychiater of psychotherapeut bezocht, dat is 10\%.\cite{1-in-10-pygy} Ik ben van mening dat emotionele volwassenheid en capaciteit zeer in waarde word onderschat. Emotioneel volwassen worden hoort minstens even goed bij de opleiding van hoe je later een succesvolle deelnemer word in de maatschapij.
            
            \bigskip
        
            \noindent Emoties, hoe je hier mee om moet gaan is niet makkelijk voor iedereen. Sommige mensen hebben ouders die het zelf niet weten. Hoe kunnen die kinderen het dan aan leren? Daarom denk ik dat dit een must have skill is voor iedereen die met kinderen om moet gaan. Het is namelijk zo dat iemand emotionele staat extreem belangrijk is voor, hun welzijn, motivatie, en uiteraards prestaties. 
            
            \bigskip
        
            \noindent Een ander groot probleem dat je hier mee helpt is een versie van generational trauma. Kinderen die niet leren omgaan met hun emoties, kunnen hun kinderen niet leren omgaan met hun emoties. Deze kinderen kunnen zelf als ouders ook niet hun kinderen leren omgaan met emoties enz. Daarom is het naar mijn mening net zo belangrijk als rekenen dat kinderen leren hoe ze met emoties om moeten gaan.
            
            \bigskip
        
            \noindent Verder is het voor kinderen heel belangrijk om sociale contacten te hebben. Kinderen moeten sociale banden hebben voor hun ontwikkeling.\cite{social-isolation-affect-a-childs-mental-health-and-development} Het missen van sociale connecties is niet alleen slecht voor hun ontwikkeling. Het zorgt er ook voor dat ze later in het leven minder vrienden hebben. Wat niet goed is voor hun welzijn in de long term. Dit is ook een eis volgens motivatietheorie. Waardoor ik vind dat het hoog op de prioriteiten van elke leeraar moet staan.

    \newpage
