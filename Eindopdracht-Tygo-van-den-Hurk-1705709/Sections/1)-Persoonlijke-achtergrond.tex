    % --------------------------------------------------------------------------
    \section{Persoonlijke achtergrond}
        \textbf{Persoonlijke achtergrond: Schets hier jouw onderwijscarrière tot nu toe en hoe je die wenst voort te zetten. Gebruik de zelfportret opdracht uit bijeenkomst 1. Welke ervaringen heb je gehad als leerling en als docent? Wat is jouw toekomstperspectief? Denk je dat je docent gaat worden/blijft? Waarom wel/niet? En binnen wat voor soort omstandigheden/welke doelgroep/welk vak?}

        \subsection{Achtergrond}
            Als ik mijn achtergrond kort zou moeten beschrijven, dan zou dat ook de manier zijn hoe ik hem zou moeten beschrijven, kort. Ik heb in mijn jaren niet veel op genomen actief. In mijn mening, ben ik door de basis- en middelbare school heen gevlogen. zonder mijn heresenen aan te hoeven zetten. Daar gebeurde op beide niet veel. De stof was makkelijk, en ik heb als kind niet veel opgelet wat de leeraar, klasgenoten, of zelfs ik deed. Ik was voornamelijk bezig met hoe ik stiekem kon strips kon maken terwijl ik aan het "rekenen" was. Het is pas nu, als ik reflecteer dat ik over wat ik nog kan herrinier kan na gaan waarom mensen dingen deden.
            
            \subsubsection{Leerling's Perspectief}
                Zoals hierboven vermeld was ik niet perse een goede student. Heel plat gezegd was ik gewoon slimmer dan dat je voor de stof hoefte te zijn en kon ik mijn weg er doorheen faken, of met minimale moeite doen. Dat maakte mij snel afgeleid en bezig met dingen bezig die ik intresant vond.

                \bigskip
        
                \noindent Er zijn wel dingen die ik me kan herineren waar ik een mening over heb. In groep 6 had ik een leeraar die kinderen deed pesten om ze te laten gedragen hoe hij wilde. Als ik was weg gedroomd deed hij altijd luid "Ik zat even niet op te leten, kwam met mijn vingers tussen de castagnetten..." een liedje van Bertus Staigerpaip te zingen, zodat iedereeen dat kon horen of dat zette hij dan aan op youtube. Een andere peste hij voor zijn kleine blaas omdat hij te vaak naar het toilet moest. Hij at krijtjes om te schuimbekke zodat ie ze je kon laten schrikken. Een nare man. Ik zou zeggen, dat is hoe het niet moet.

                \bigskip
        
                \noindent Jaar daar na, in groep 7, kreeg ik juf Ellie en juf Inge. Dat was dan weer een voorbeeld over hoe je het wel moest doen. Hun waren leeraren die kinderen begrepen en het beste er uit wilde halen. In plaats van mij te pesten omdat ik snel afgeleid was waren ze er juist voor om mijn concentratie op te bouwen. Ze lieten me steeds langer concentren. Eerst maar 5 minuten aan een stuk. Dan moest ik 5 minuten rekenen. Endaarna mocht ik 5 minuten tekenen. Ik vond dat best eerlijk 50/50. Dan bouwede ze het langzaam op naar 20 minuten rekenen. 5 minuten tekenen. Dit heet progressive concentration training heb ik zojuist op gezocht en dit, is in mijn mening wel een manier hoe je met kinderen moet omgaan, hoe je leeraar kan zijn.
                
            \subsubsection{Leeraar's Perspectief}
                Ik heb nog niet veel ervaring als een leeraar. Ik heb hockey les gegeven daar had ik veel plezier aan. Het probleem is dat ik alleen me snel in chaos liet trekken. Ik ben zelf van nature veel random, chaotisch, en impulsief. Dat is de ADHD, maar dat maakte het niet makkelijker om met 13 kleine kinderen te moeten dealen. Dus dat is iets waar ik aan het werken ben.
                
                \bigskip
        
                \noindent Ik heb ook stof uit gelegd aan mede leerlingen voordat ik op uni ook zelf begon te struggelen. Hoewel dat niet altijd goed ging, en het kwartje niet altijd meteen viel merkte ik wel dat de efectiviteit wel flink verschilde per persoon. Ik denk dat dit is omdat de manier hoe ik begreep niet altijd logisch was voor andere en andersom. Dat is natuurlijk omdat ieder mens anders denkt en een ander brein heeft. Maar dat bracht mij wel op een idee voor hoe ik dit in de toekomst wil zien.
                
        % — — — — — — — — — — — — — — — — — — — — — — — — — — — — — — — — — — — 
        \subsection{Toekomst Visie}
            Dat brengt mij op mijn visie voor de toekomst. Ik had, en heb nog steeds struggles met mijn brein. Mijn brein is wat je heel "bijzonder" zou noemen. Met ADHD, Dyslextie, en wat op dit moment onderzocht word waarschijnlijk autisme. Dat betekend dat mijn brein op een andere manier is opgebouwd, denkt, functioneer, groeit, verwerkt, of stof op neemt\cite{neurodivergent}. 
        
            \bigskip
        
            \noindent Dat is niet perse slecht, soms is dit een super sterke kant. Denk aan bijvoorbeeld een kenmerk van ADHD genaamd "hyperfocus"\cite{hyperfocus} veel van ons die ADHD hebben beschikken hierover. Dit is een soort focus waar je in kan stappen — technisch gezien val je hier meer in — waarin je de hele wereld vergeet en je brein alleen maar die activiteit die je aan het doen bent kan ademen. Er zijn vele dagen — zeker nu dat ik op mezelf woon — dat ik lunch, en avondeten vergeet. Dat is niet mijn bedoeling, maar calculus klikte en kwam in de flow en voor ik het wist was het 10 uur s'avonds.
        
            \bigskip
        
            \noindent Mijn punt dat ik wilde maken hier is omdat iedereen anders is, iedereen ook andere behoefte heeft. Mensen met ADHD, autisme, dyslextie, en zelf queerness hebben gewoon een andere brein dan neurotypische mensen\cite{ADHD-Neurobiologie}\cite{Autisme}\cite{Dyslextie-breinen}\cite{LGBT-vs-CISHET-breinen}. Daarmee hebben zij ook andere behoefte\cite{ADHD-Bijkomende-problematiek}\cite{ADHD-behoeftes}. Mijn toekomst visie is dat neurodivergent breinen ook het onderwijs krijgen dat ze nodig hebben en dat ondanks dat ze andere brein structuren hebben ze niet perse tweede rangs onderwijs hoeven te hebben. Maar hier vertel ik in de sectie "Mijn Ideal Les" meer over.
        
        % — — — — — — — — — — — — — — — — — — — — — — — — — — — — — — — — — — — 
        \subsection{Terug koppeling on mijn Zelf-Portret}
            Om terug te komen op mijn zelf potret. Ik weet nog steeds niet of ik leeraar wil worden. Wat ik wel weet is dat als ik leeraar word. Ik graag wil zorgen dat het onderwijs inclusiever word. Er kan nog zo veel gedaan worden om het onderwijs beter te maken voor iedereen. En of ik voor de klas wil staan en op die manier die verandering wil maken, dat weet ik nog niet. Maar dat die verandering op een bepaalde manier moet komen — one way, or another — dat weet ik wel. Dat voel ik.

    \newpage
